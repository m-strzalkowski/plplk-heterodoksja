\documentclass[12pt]{aghdpl}
\graphicspath{ {./images/}{.} }
\usepackage[inkscapelatex=false]{svg}
% \documentclass[language=en,11pt]{aghdpl}  % praca w języku angielskim

%---------------------------------------------------------------------------

\author{Mateusz Strzałkowski}
\shortauthor{M. Strzałkowski}


\titlePL{\huge{ Heterodoksyjnie proceduralny język programowania i jego kompilator.}}
\titleEN{\large{Heterodoxically procedural programing language and its compiler.}}


\shorttitlePL{Heterodoksyjnie proceduralny język programowania i jego kompilator.} % skrócona wersja tytułu jeśli jest bardzo długi
\shorttitleEN{Heterodoxically procedural programing language and its compiler.}


% Dopuszczalne wartości[1,2]:
% * "Projekt dyplomowy" - na koniec studiów I stopnia
% * "Praca dyplomowa" - na koniec studiów II stopnia
% [1] Zasady dyplomowania w roku akademickim 2020/2021 (Decyzja Dziekana WEAIiIB nr 16/2020 z dnia 9 grudnia 2020 roku)
% [2] Załącznik nr 1a) do Decyzji nr 16/2020 Dziekana Wydziału EAIiIB z dnia 09 grudnia 2020 r.
\thesistype{Projekt dyplomowy}
%\thesistype{Master of Science Thesis}

\supervisor{dr Dariusz Pałka}

\degreeprogramme{Informatyka i systemy inteligentne}
%\degreeprogramme{Automatics and Robotics}

\date{2024}

%\department{Katedra Informatyki Stosowanej}
%\department{Department of Applied Computer Science}

\faculty{Wydział Elektrotechniki, Automatyki, Informatyki i Inżynierii Biomedycznej}
%\faculty{Faculty of Electrical Engineering, Automatics, Computer Science and Biomedical Engineering}

\acknowledgements{Serdecznie dziękuję promotorowi, panu Terence Parrowi (za antlr) oraz autorom projektu LLVM}

\definecolor{mygreen}{rgb}{0,0.6,0}
\definecolor{mygray}{rgb}{0.5,0.5,0.5}
\definecolor{mymauve}{rgb}{0.58,0,0.82}
\newcommand{\blue}[1]{\textcolor{blue}{#1}}
\newcommand{\red}[1]{\textcolor{red}{#1}}
\newcommand{\up}{\textasciicircum}
\lstset{ %
  backgroundcolor=\color{white},   % choose the background color
  basicstyle=\footnotesize,        % size of fonts used for the code
  breaklines=true,                 % automatic line breaking only at whitespace
  captionpos=b,                    % sets the caption-position to bottom
  %commentstyle=\color{mygreen},    % comment style
  escapeinside={\%*}{*)},          % if you want to add LaTeX within your code
  %keywordstyle=\color{blue},       % keyword style
  %stringstyle=\color{mymauve},     % string literal style
}

\begin{document}

	\titlepages
	\RedefinePlainStyle
	
	\setcounter{tocdepth}{2}
	\tableofcontents
	\clearpage
	
	\chapter{Wprowadzenie}
\label{cha:wstep}

Wobec mnogości i różnorodności dostępnych obecnie języków programowania, zaskoczeniem może się wydać, że istnieją pewne koncepcje językowe nigdy nie zrealizowane, bądź historycznie wypróbowane, a dziś nieobecne. W niniejszej pracy rozwinięty zostanie nowy proceduralny język programowania, wykorzystujący i łączący kilka interesujących form lingwistycznych.
Urzeczywistni się ten język poprzez program jego automatycznego translatora dla maszyn cyfrowych. Możliwość i poziom trudności skonstruowania praktycznie użytecznego kompilatora nowego języka, dysponując ograniczonymi zasobami, lecz korzystając ze współczesnych narzędzi w sposób poparty ugruntowanymi osiągnięciami lingwistyki formalnej, będzie również kluczowym przedmiotem badań.

%---------------------------------------------------------------------------

\section{Czy nowe języki programowania są potrzebne?}
Języki ,,wysokiego poziomu'', które dziś odpowiadają powszechnemu pojęciu o ,,językach programowania''pojawiły się w latach pięćdziesiątych ubiegłego wieku.\cite[s.~13]{DRAGON_BOOK} Wydawać by się mogło, że od tego czasu, problem, na jaki stanowiły odpowiedź - zapewnienie zrozumiałego zarówno dla człowieka jak i maszyny opisu algorytmu, po tylu dekadach doczekał się zadowalających rozwiązań - innymi słowy zaproponowano i przetestowano większość wartych uwagi możliwości i dostępne obecnie języki programowania są co najmniej wystarczające i pole do usprawnień pozostaje niewielkie.

Zdaje się mieć to odzwierciedlenie w spadającej w ostatnich latach liczbie języków nowych - takich, które zyskały publiczne uznanie i grono użytkowników.\cite{valverde2015} Powodem może być pewna dojrzałość wypracowana przez najpopularniejsze obecnie języki oraz osiągnięcie przez nie takiego poziomu abstrakcji, na którym problemy niegdyś modelowane przez specjalnie określone struktury syntaktyczne, oddać można za pomocą istniejącej, dużo ogólniejszej składni.

Przykładem niech będzie obsługa dostępu do plików w języku PL/I popularnym w latach 70, gdzie istniał dedykowany aparat w składniowy, zawierający np. osobne słowo kluczowe FILE\cite[s.~96]{plif}. We współczesnym C++, czy Javie, w specyfikacji języka próżno szukać przeznaczonych do tego konstrukcji. Operacje na plikach realizowane są poprzez funkcjonalności bibliotek standardowych i nie wymagają modyfikacji języka.

Wydawać by się mogło, że po osiągnięciu pewnego poziomu abstrakcji syntaktycznej (w szczególności typów parametryzowanych), innowacje mogą ograniczyć się do rozszerzania leksykonu języka, nie modyfikując składni. Poniższy przykład, wzięty z bardzo obecnie rozpowszechnionej praktyki programistycznej ma za zadanie podważyć to twierdzenie i zwrócić uwagę na złożoność relacji pomiędzy składnią a leksykonem języka.

Napisana w Javie biblioteka Spring nie jest pierwszym i z pewnością nie ostatnim podejściem do stworzenia ekosystemu ułatwiającego tworzenie serwerów i złożonych aplikacji sieciowych. Najtrafniejszym jej określeniem pozostanie jednak ,,framework'', ponieważ nie stanowi ona jedynie zbioru procedur gotowych do użycia, jak biblioteka, a określa również szkielet aplikacji i do pewnego stopnia redefiniuje sposób pisania programu, teoretycznie wciąż w języku Java. Niech zademonstruje to poniższy wycinek:
\begin{lstlisting}[language=java]
@Auth
@RestController
public class CarResource{
    @Autowired private CarService service;

    @Permission(Perm.CarR)
    @GetMapping("/cars")
    GetCarsResponse getCars(GetCarRequest request){
        return service.availableCars(getFilters(request));
    }
    ...
    @Permission(Perm.CarRW)
    @PostMapping("/cars")
    ReserveCarResponse reserveCar(String){...}
}
\end{lstlisting}
%spr - Spring vs Java EE
Pierwszą obserwacją niech będzie, że człowiek zaznajomiony jedynie ze specyfikacją języka Java marne ma szanse zrozumienia powyższego wycinka programu. Jeśli orientuje się w zagadnieniach sieciowych, zna pojęcie REST API i programowanie zorientowane obiektowo, może zrozumieć już bardzo wiele, potencjalnie cały zamysł programu, umknąć mu jednak mogą kluczowe szczegóły. Na przykład, zakładając, że reserveCar dokonuje modyfikacji w bazie danych, ów programista mógłby zacząć zadawać pytania o zachowanie systemu w wypadku błędu wewnątrz procedury rezerwującej samochód. Jeśli nie jest przeszkolonym użytkownikiem Springa, nie ma sposobu aby domyślił się, że pewna adnotacja nad klasą, której metodę się wywołuje, odpowiada za „opakowanie” jej nową transakcją baozodanową (jeśli takowej wcześniej nie rozpoczęto) i odwinięcie jej w wypadku wystąpienia wyjątku podczas wykonania kodu wewnątrz, a jest to szczegół o potencjalnie kluczowym znaczeniu dla analizy programu.

Do jakiego stopnia jest zatem ów program napisany w Javie, do jakiego stopnia w Springu? Czy Spring jest jeszcze biblioteką, frameworkiem, czy może raczej osobnym językiem? Fakt, że ów język pozostaje podzbiorem Javy i jest wciąż tłumaczony przez jej kompilator ma jednak swoją cenę. 

Java jest w założeniach językiem kompilowanym i statycznie typowanym. Zapewniać ma to bezpieczeństwo typów, wymuszane podczas tłumaczenia programu. Poniższa linia:
\begin{lstlisting}
    CarService service = new CarService(carRepository);
\end{lstlisting}
nie powinna pozostawiać pola dla wielu niespodzianek podczas wykonania, gdy zostanie pomyślnie skompilowana.
Jeśli programista zapomni dopisać klasę CarSevice, bądź ją zaimportować aby była widoczna w tym pliku (jednostce translacji), od razu otrzyma błąd kompilacji.
Konstrukcja springowa będąca jej odpowiednikiem:
\begin{lstlisting}
@Autowired private CarService service;
\end{lstlisting}
nie zapewnia nawet tych prostych gwarancji. W podstawowym przypadku, gdy CarService fizycznie nie istnieje, faktycznie, tak samo nazwa nie zostanie odnaleziona w zakresie. Gdyby jednak wstrzykiwany obiekt był interfejsem np.,
\begin{lstlisting}
@Autowired ICarService service; 
(w innym miejscu:)CarService implements ICarService
\end{lstlisting}
albo typem parametryzowanym (w nomenklaturze Javy - generycznym)
\begin{lstlisting}
@Autowired PageableService<Car> service; 
(w innym miejscu: )CarService extends PageableService<Car>,
\end{lstlisting}
program mógłby się skompilować i dopiero przy jego uruchomieniu wystąpiłby błąd - maszyneria Springa mogłaby nie móc odnaleźć (np. z powodu błędnej konfiguracji) odpowiedniej klasy będącej podtypem interfejsu lub typu parametryzowanego.

Język Springa przestaje być zatem statycznie typowany, traci gwarancję sprawdzenia typów podczas przekładu. Dzieje się tak dlatego, że Spring przy całej swej złożoności pozostaje na technicznym poziomie biblioteką. Nie jest w stanie wprowadzić własnych ograniczeń do sprawdzania typów podczas kompilacji, ponieważ jego procedury same są częścią tłumaczonego programu. 

Przykład ten nie ma na celu krytyki niczego, w końcu istnieje wiele języków dynamicznie typowanych, niektóre bardzo popularne (np. Python), pokazuje,  że języki programowania, powstają zupełnie organicznie również obecnie i to w wielkich ilościach, nie nosząc wszak nazwy „języków” - wystarczy zacząć liczyć frameworki, z których każdy podlega podobnym zjawiskom.

Widzimy też, że podejście biegunowo odmienne od prezentowanego przez niektóre języki ubiegłych dziesięcioleci, <z dedykowaną składnią dla wielu powszechnych funkcjonalności>, też ma swoje wady a mianowicie brak możliwości korzystania z mechanizmów kompilatora do kontroli konkretnych, praktycznie użytecznych ograniczeń podczas tłumaczenia.

Do czasu wprowadzenia w kompilatorach wielkich języków ogólnego przeznaczenia mechanizmów integracji z bibliotekami (o ile kiedykolwiek to nastąpi), pozostają otwarte możliwości wykorzystania bardziej wyważonego podejścia i tworzenia języków w większym stopniu uszczegółowionych i przez to umożliwiających statyczne kontrole dla konkretnych ograniczeń. Uważam to za pierwszy powód, dla którego wciąż można myśleć o nowych językach programowania.
Pozostaje jednak problem, jak pomysł na język przełożyć na działający kompilator.

\section{ O trudności pisania kompilatorów}
Sztuka automatycznej translacji języków bezkontekstowych, (które są teoretycznym modelem dla większości języków wysokiego poziomu), rozwinęła się do tego stopnia, że poza niszowymi zastosowaniami, jak pisanie sterowników, czy niektórych fragmentów kodu systemów operacyjnych, gdzie albo trzeba zarządzać ustawieniami maszyny cyfrowej poprzez wykonanie specjalnych instrukcji (np. ustawianie wektora przerwań), albo wyzyskać własności konkretnej architektury sprzętowej\cite{kernel_exception_handling},programowanie w językach niższego poziomu (asemblerach) wymarło niemal zupełnie, ponieważ kompilatory optymalizują kod tak dobrze jak wykwalifikowany programista asemblerowy (którego niełatwo dziś znaleźć, a programista niewykwalifikowany piszący w asemblerze napisze wielkim nakładem pracy kod wolniejszy i potencjalnie zawierający błędy).[] Sztandarowym przykładem efektywnego kompilatora optymalizującego jest gcc (GNU C compiler).%[zob. Jądro Linuksa w ogóle], 

Język C, używany wciąż od ponad pół wieku, zajmuje w drzewie filogenetycznym języków programowanie miejsce podobne łacinie w rodzinie języków indoeuropejskich. Tak jak z języka Rzymian wywodzą się języki romańskie, tak C, zyskawszy wpierw wielką popularność, zostało w latach dziewięćdziesiątych rozwinięte twórczo na wiele sposobów, będących głównie próbami stworzenia na jego podstawie języków zorientowanych obiektowo - wymienić można C++, Objective C, a potem już pośrednio np. javę i C\#. 
Oprócz tych „wielkich języków”, powstało jednak wiele bardziej wyspecjalizowanych do zastosowań domenowych, np. GLSL (do obliczeń na kartach graficznych), czy prostych narzędzi, np. awk, którego język jest wszak również formalnie językiem programowania.

Porównałem języki programowania z naturalnymi nieprzypadkowo, ponieważ te sztuczne również wydają się posiadać pewien cykl życia - kształtowania się, zyskiwania popularności i stopniowego wymierania. Jednak o ile żywot języków naturalnych kształtowany jest złożonymi procesami cywilizacji ludzkiej, będącymi zazwyczaj poza kontrolą jednostek, o tyle w wypadku języków sztucznych, jest wynikiem celowej aktywności umysłowej, a każdy kolejny język jest w pewnym sensie ulepszeniem względem poprzednich, zastosowaniem wyciągniętych z ich użytkowania wniosków, poprawą niedoskonałości.\cite{valverde2015}

O ile wymieniona, "obiektowa" część drzewa potomków C, zdaje się podlegać ciągle ulepszeniom i stosowaniem wniosków poprzez stopniowe zmiany w kolejnych wersjach języka (Np. dodanie typów generycznych do Javy), o tyle w ostatnich latach zachodzi ciekawe zjawisko, które można by określić mianem wyciąganiem na nowo wniosków z imperatywnego C.

Bowiem, język C nie wymarł wraz z pojawieniem się jego następców, nie wszedł nawet w typową schyłkową fazę rozwoju języka, współegzystującego długo wraz z bezpośrednimi następcami głównie z powodu mnogości programów w nim napisanych (i koniecznych do utrzymania). Ku pewnemu zaskoczeniu głosicieli obiektowego podejścia, pozostała grupa zastosowań, gdzie wciąż używa się zwyczajnego C. Nie można tego przypisywać wyłącznie wyróżnionej jego pozycji np. ze względu na to, że jądro Linuksa jest w nim napisane i wszystkie programy linuksowe „zejść” muszą w końcu do ABI (binarnego interfejsu) C. W pewnych zastosowaniach C może pozostać stosowany niemal dowolnie długo - przede wszystkim w oprogramowaniu wbudowanym. Jego względna prostota, wciąż najlepsza wydajność oraz możliwość przeprowadzania niskopoziomowych operacji czyniły go przez długie lata niezastąpionym. Jednakże niektóre jego cechy, te same, które decydują o jego sukcesie, okazują się niekiedy kłopotliwe. Ręczne zarządzanie pamięcią na stercie i odnoszenie się do niej poprzez wskaźniki zawsze było powodem wielu błędów, a wraz z brakiem sprawdzania granic tablic często źródłem podatności bezpieczeństwa poprzez całą klasę ataków przy pomocy przepełnienia bufora itp. 

Długo nie istniała na rynku alternatywa dla C. C++ był prawie tak samo szybki (gdy użyty poprawnie), lecz obiektowy, co nie było dla wszystkich akceptowalne. W pewnym jednak momencie, w drugiej dekadzie XXI wieku, do języków potomnych względem C zaczęły z wolna przesączać się koncepty mające źródła w zupełnie innych drzewach rozwoju języków i paradygmatach, w szczególności funkcyjnym. F\# pokazał, że język funkcyjny może być wydajny (choć wciąż zbyt ezoteryczny dla zwykłego programisty), posunęła się również myśl w zakresie systemów typów, czerpiąc miedzy innymi z języków pokroju ML. Pojawiły się wpierw typy parametryzowane (generyczne w Javie, szablony w C++), alternatywy typów, (operator | w pythonie, koncept Optional), do pewnego stopnia inferencja typów w miejsce konsekwentnego niegdyś ich syntetyzowania. Innowacje te rozprzestrzeniły się po gałęziach potomków C, aż można zaryzykować stwierdzenie, że wielu ludzi zaczęło się zastanawiać, jak wyglądałby sam ascetyczny C, gdyby przepisać go na nowo, uwzględniwszy te wszystkie nowe wnioski i pomysły.

W przeciągu pięciu dziesięcioleci zaszła ponadto jedna zmiana o kluczowym znaczeniu - zwielokrotniła się (dalej przestrzegając prawa Moore'a) moc maszyn cyfrowych i w konsekwencji zelżała presja na wymagania obliczeniowe i pamięciowe samych kompilatorów. Dziś dość abstrakcyjne mogą się wydawać relacje z epoki mówiące o tym, że kompilator wraz ze środowiskiem uruchomieniowym zajmował 86 kilobajtów pamięci operacyjnej, pozostawiając 14 kB na program studenta na jednym z amerykańskich uniwersytetów w latach 70[wikiPL/M??]. Kompilacja była jednym z bardziej wymagających obliczeniowo zadań w tamtej epoce, dlatego tak wielki nacisk kładziono na jak najefektywniejszą implementację kompilatorów, co widać wciąż w bodaj najsłynniejszego podręcznika w tym zakresie\cite{DRAGON_BOOK}.
Ubocznym, lecz nieuniknionym skutkiem takiego nacisku na wydajność, była konieczność rezygnacji z niektórych, zbyt wymagających obliczeniowo, lub pamięciowo metod analizy, a w szerszym kontekście, zaniechanie wprowadzania i rozwoju takich elementów językowych, których tłumaczenie by ich wymagało.

Tymczasem jednak kompilatory wielu nowszych języków wyższego poziomu nie były tak restrykcyjne. Każdy, kto kompilował większy projekt w Javie, wie dobrze ile może zająć to czasu i pamięci. Popularna implementacja JAVA JDK (de facto bibliotekii standardowej) ma wymagania sprzętowe do jej kompilacji porównywalne ze współczesną grą komputerową.\cite{jdk_build_requirements}

Czynniki te razem, zdaniem autora, złożyły się na powstanie nowej grupy języków, będących niejako (chociaż nigdy zupełnie) reinterpretacjami C, pozostając wciąż na podobnym poziomie abstrakcji - Go, Rust, Carbon,Zig, czy mniej znany V.
Z tych pięciu wymienionych jeden transpilowany jest do C - V\cite{vlang_repo}, jeden jest posiada pełny kompilator w tradycyjnym rozumieniu - Go\cite{go_faq}, a trzy pozostałe korzystają z projektu LLVM\cite{rust_repo}\cite{carbon_repo}\cite{zig_repo}. Żeby zrozumieć, czym on jest i dlaczego stanowi poważne ułatwienie w konstrukcji kompilatorów, należy sobie uświadomić, jak złożonymi programami są automatyczne translatory sztucznych języków do kodu wykonywalnego. 
\clearpage

\section{Zarys przebiegu kompilacji}
Diagram poniżej przedstawia klasyczną, ogólną architekturę kompilatora.

\begin{figure}[H]
    \centering
    \includegraphics[height=0.8\linewidth]{images/wstep/fazy_kompilacji.png_bin.png}
    \caption{Fazy kompilacji, zaczerpnięte ze str. 4 podręcznika\cite{DRAGON_BOOK}}
\end{figure}

 
Automatyczny translator musi rozłożyć zdanie w języku wejściowym, „zrozumieć” je, czyli przekształcić na wygodną dla niego reprezentację, aby z owego modelu złożyć reprezentację w języku docelowym. Wychodzi się od jednego napisu, dokonuje jego rozbioru, analizy, przeprowadza pewne operacje i sprawdzenia na odpowiednio skonstruowanych reprezentacjach głębszych konceptów wyrażanych językiem i potem „zwija” się je z powrotem do tekstu, tym razem w języku docelowym. Część analityczną nazywa się „front endem” , a syntetyczną „back endem”.\cite[s.~4]{DRAGON_BOOK}

Z dobrych powodów, sprowadzających się do zapewnienia porządku i możności rozumienia przez programistę, cały proces podzielony jest na fazy, odzwierciedlające się w warstwowej strukturze aplikacji, tak, że można wizualizować kompilator jak na powyższym rysunku – jako łańcuch spiętych ze sobą komponentów przekazujących sobie kolejno przekształcane formy programu. Ma to pewne odniesienie do człowieka – my również prawdopodobnie posiadamy obwody w mózgu, wyspecjalizowane w rozpoznawaniu znaków, liter i wyodrębnianiu z nich słów, przekazując sygnały do obwodów wyższego poziomu, układających słowa w sensy i zdania, a równocześnie odbywa się sięganie do pamięci, by przywołać znaczenia poszczególnych leksemów. Podobnie w kompilatorze analizator leksykalny zasila analizator składniowy (parser) i pisze się do tablicy symboli.[*]
Niezależnie od tego, jak w rzeczywistości umysły ludzkie rozkładają zdania w językach naturalnych, lingwiści przywykli rozrysowywać strukturę zdań w postaci drzew składniowych (z podmiotem, orzeczeniem, przydawkami i innymi częściami zdania). Niektórzy z nich, zamiast gramatyk opisowych, w połowie XX w. w rozwinęli pojęcie „gramatyki generatywnej”, będącej zbiorem produkcji, ściśle określającej, jakie wyprowadzenia zdań są dozwolone.\cite{Chomsky1956} Podejście to, choć okazało się niewystarczające dla języków naturalnych, zaadoptowano[*], gdy zaczęto tworzyć sztuczne naśladownictwa naturalnych języków, zawężone i sformalizowane, a przez to możliwe do automatycznego tłumaczenia, w szczególności na kod maszynowy. 

%nagłówek tutaj jakiś
Zagadnienia analizy składniowej,  cytując Niklausa Wirtha, twórcę Pascala „stanowią jedynie część problemów translacji i to w gruncie rzeczy stosunkowo niewielką. Jest to również część dająca się najłatwiej sformalizować, co m.in. pozwala na reprezentację składni języka w postaci dość regularnej struktury danych. Znacznie większe trudności napotykamy przy próbie formalizacji semantyki języka […]” [N. Wirth Algorytmy + Struktury Danych = Programy, str. 317]

O ile do dwóch pierwszych etapów rozwinięto ścisłą, użyteczną i dość uniwersalną teorię, o tyle nie wypracowano tak rozbudowanego aparatu semantycznej, w innych aspektach niż systemy typów [TAPL – książka], nie jest ona szeroko opisywana i programista kompilatora polegać musi w wielkiej części na własnej inwencji. Dzieje się tak dlatego, że owa środkowa część kompilatora, najbardziej odpowiada idiosynkrazjom konkretnego języka. W tym etapie, przekształcamy program od drzewa rozbioru, zależnego prawie wyłącznie od formalizmu gramatyki, do pośredniej postaci, bliższej maszynie, przydatnej przy optymalizacjach, rządzonymi równie potężnymi formalizmami matematycznymi. Znajdująca się pomiędzy tymi dwoma terytoriami poddanymi ścisłym badaniom warstwa pośrednia przetwarza najbardziej analityczną postać programu, gdzie wszystkie jego językowe własności zostają rozkodowane z tekstu źródłowego i przechowywane jawnie w strukturach danych, odpowiadających bezpośrednio konceptom języka. Tam zakres leksykalny reprezentowany jest określoną strukturą zawierającą listę symboli, typ danej zmiennej jest osobnym  obiektem, klasa przechowywania (static/automatic z C) pewną wartością wyliczeniową. Niechybnie musi zatem ta faza charakteryzować się wielką zmiennością, w zależności od języka – translator języka obiektowego będzie zawierał reprezentacje klasy i będzie ona wręcz centralna dla modelu danych, w funkcyjnym języku przeciwnie, reprezentacje funkcji, ich typów i domknięć zajmą poczesne miejsce. Zakodować też musimy tam wszystkie drobne szczegóły i reguły semantyczne jakie język posiada. Dlatego obok coraz potężniejszych teorii typów, oraz przydatnych raczej na początku tej fazy gramatyk atrybutywnych, nie istnieje ogólnie przyjęta teoria, ani nawet praktyczny przepis na realizację analizy semantycznej.
%<akapit powyżej- zupełne śmiecie – poprawić>

Inaczej ma się rzecz z optymalizacjami. Nie można ich pominąć – to, że pierwszy kompilator Fortranu (jednocześnie pierwszy w historii kompilator języka wysokiego poziomu), w ogóle został przyjęty przez środowisko informatyczne, umożliwiło zapewnienie, że tworzy kod maszynowy prawie zawsze tak samo szybki, jak programista asemblerowy. Największą zaletą w oczach współczesnych, mogła nawet nie być możliwość prawie dosłownego przepisywania wyrażeń matematycznych, a efektywne rozplanowywanie wielokrotnych pętli przy operacjach macierzowych (na tablicach z wieloma indeksami), która to czynność zajmowała wcześniej programistom ogromne ilości czasu i była podatna na błędy, nie chciano się jednak godzić na żadne interpretery, czy prymitywne rozwiązania, drżąc przed roztrwonieniem przez nie wątłych zasobów ówczesnych komputerów[FORTRAN REPORT].
Odnoszę się do tych zamierzchłych dyskusji, aby uzmysłowić, jak centralnym zagadnieniem kompilacji są optymalizacje i bez nich, bez produkcji równie efektywnego kodu, jak byłoby to możliwe ręcznie w języku docelowym, nie można mówić o prawdziwie użytecznym  kompilatorze.
<też do przepisania>

Problemy analizy przepływu, przenoszenia niezmiennego kodu na zewnątrz pętli, detekcji kodu martwego, efektywnego traktowania rekursji ogonowej i niemożliwej do wymienienia tutaj rzesze optymalizacji wynalezionych na przestrzeni ostatnich dziesięcioleci, są naturalnie trudne, zaryzykuję stwierdzenie, że wielokroć trudniejsze od parsingu, bo wymagają bardzo ścisłej wiedzy matematycznej, nie są też tak często przedmiotem kursów na uczelniach. I te zagadnienia mogą się wydać relatywnie zachęcające, gdyż dalej w procesie translacji schodzić trzeba w głąb, aż do fizycznej maszyny – problem przydziału rejestrów nie dość, że jest NP-trudny[wiki/register allocation → Chaitin et al 1981], to jeszcze zależny od konkretnych technikaliów maszyny cyfrowej. W niektórych architekturach pewne rejestry mogą mieć szczególne ograniczenia, poszczególne instrukcje właściwe tylko sobie zachowania. Na przykład, na znanej nam wszystkim 32 bitowej architekturze Intela, dwa operatory całkowite znane z C – dzielenia i modulo tłumaczone są na tę samą instrukcję, IDIV bowiem pozostawia iloraz w rejestrze eax a resztę w edx – o wszystkich takich szczegółach należy pamiętać i uwzględnić w algorytmach. Podczas generacji kodu trzeba również, oprócz realizacji oczywistych schematów translacji dla warunków i pętli, tablic skoków dla switch, budować ramki stosu i zapisywać niektóre rejestry, zależnie od architektury i konwencji przy wywołaniach, a jeśli umieściło się w języku wyjątki podobne tym z C++, należy wygenerować zupełnie nietrywialny kod do ich obsługi. W całości składa się to na niezwykle pracochłonne zadanie, które wszak należy powtórzyć, prawie zupełnie niezależnie dla każdej architektury docelowej.

Nie jest to wszak koniec procesu translacji, nie ostatni komponent na rycinie zamieszczonej powyżej – mając nawet wygenerowany wykonywalny kod obiektowy, nie możemy załadować go verbatim do pamięci maszyny, ustawić na konsoli operatora wartości licznika rozkazów i nacisnąć guzika – jak ongiś się zdarzało – kod trzeba jeszcze skonsolidować i przygotować do interakcji z programem ładującym systemu operacyjnego, aby mógł on przekształcić nasz obraz binarny w „żywy” obraz procesu. Jeśli ktoś sądzi, że jest to jedynie kwestia przekazywania odpowiednich flag do GNU ld (albo odpowiednika w danych systemie operacyjnym), niech przeczyta poniższy wycinek i policzy ile zna osób wiedzących w ogóle co to jest skrypt linkera, a co dopiero potrafi takowy zrozumieć.
%[https://gist.github.com/avdgrinten/40e1e8615f46f2fd82226e2ed98d381d - Default LD linker script (elf_x86_64 on Linux) ]

\begin{lstlisting}[
    basicstyle=\footnotesize, %or \small or \footnotesize etc.
]
.plt            : { *(.plt) *(.iplt) }
.plt.got        : { *(.plt.got) }
.plt.bnd        : { *(.plt.bnd) }
  .text           :
  {
    *(.text.unlikely .text.*_unlikely .text.unlikely.*)
    *(.text.exit .text.exit.*)
    *(.text.startup .text.startup.*)
    *(.text.hot .text.hot.*)
    *(.text .stub .text.* .gnu.linkonce.t.*)
    /* .gnu.warning sections are handled specially by elf32.em.  */
    *(.gnu.warning)
  }
  .fini           :
  {
    KEEP (*(SORT_NONE(.fini)))
  }
  PROVIDE (__etext = .);
  PROVIDE (_etext = .);
  PROVIDE (etext = .);
  .rodata         : { *(.rodata .rodata.* .gnu.linkonce.r.*) }
  .rodata1        : { *(.rodata1) }
  .eh_frame_hdr : { *(.eh_frame_hdr) *(.eh_frame_entry .eh_frame_entry.*) }
  .eh_frame       : ONLY_IF_RO { KEEP (*(.eh_frame)) *(.eh_frame.*) }
  .gcc_except_table   : ONLY_IF_RO { *(.gcc_except_table
  .gcc_except_table.*) }
  .gnu_extab   : ONLY_IF_RO { *(.gnu_extab*) }
  /* These sections are generated by the Sun/Oracle C++ compiler.  */
  .exception_ranges   : ONLY_IF_RO { *(.exception_ranges
  .exception_ranges*) }
  /* Adjust the address for the data segment.  We want to adjust up to
     the same address within the page on the next page up.  */
  . = DATA_SEGMENT_ALIGN (CONSTANT (MAXPAGESIZE), CONSTANT (COMMONPAGESIZE));
  /* Exception handling  */
  .eh_frame       : ONLY_IF_RW { KEEP (*(.eh_frame)) *(.eh_frame.*) }
  .gnu_extab      : ONLY_IF_RW { *(.gnu_extab) }
  .gcc_except_table   : ONLY_IF_RW { *(.gcc_except_table .gcc_except_table.*) }
  .exception_ranges   : ONLY_IF_RW { *(.exception_ranges .exception_ranges*) }
  /* Thread Local Storage sections  */
  .tdata	  : { *(.tdata .tdata.* .gnu.linkonce.td.*) }
  .tbss		  : { *(.tbss .tbss.* .gnu.linkonce.tb.*) *(.tcommon) }
\end{lstlisting}

Potrzebę rozumienia uwydatniają komentarze do osobnych sekcji, które kompilatory C++ generować muszą specjalnie po to, by działał mechanizm wyjątków (np. .gnu\_extab), czy statycznej inicjalizacji w tym języku (.init/.fini/.ctors/.init\_array). To drugie pokazuje, że nawet tak głęboko ukryte szczegóły implementacyjne mają wpływ na język programowania na najwyższym poziomie – bo sposób działania konsolidatora i przetwarzania przez niego tych specjalnych sekcji powoduje w C++ realne i kłopotliwe zjawisko zjawisko zwane „Static initialization order fiasco”\cite{siof}. (Nie da się wymusić deterministycznego porządku wykonywania statycznych inicjalizatorów znajdujących się w różnych plikach), które skutkuje tym, że obecnie odradza się stosowanie tego mechanizmu w ogóle\cite{google_cpp_guidelines_static}

Pozwoliłem sobie na ów korowód szczegółów, aby umotywować, dlaczego tak wiele języków odeszło od generowania kodu dla fizycznych maszyn, na rzecz używania maszyn wirtualnych, albo w ogóle intepreterów – pozwala to relatywnie zmniejszyć złożoność problemów generacji kodu i osiągnąć większą przenośność, jeśli maszyna wirtualna jest pisana np. w C, którego kompilatory istnieją na chyba wszystkich powszechnych architekturach.

Prócz tego chciałem pokazać, jak rozległym tematem jest kompilacja i że w klasycznym ujęciu, obejmującym wszystkie wymienione fazy [DRAGON BOOK, Budowa kompilatorów, znaleźć stronę], znajduje się poza zasięgiem jednego człowieka, a tworzenie praktycznie użytecznych języków w celu badania ich własności, zadaniem niewykonalnym. Dość powiedzieć, że wedle raportu projektowego, napisanie wspomnianego już pierwszego kompilatora Fortranu, po fakcie wyceniono na 18 roboczolat (!), rozłożonych na kilku programistów w ciągu dwóch i pół roku.\cite{FORTRAN_AUTOMATIC_CODING_SYSTEM}

Niklaus Wirth, również wspomina tworzenie kompilatora Pascala, jako grupowe, wielomiesięczne przedsięwzięcie, prowadzone do tego technikami, zapewne niedostępnymi ordynarnym programistom. (Po zaprojektowaniu języka Pascal, napisano jego pierwszy kompilator w nim samym i przetłumaczono go ręcznie na asembler komputera IBM 704).\cite{Wirth_recollections_Pascal} Spoglądając w bardziej współczesne czasy, można ocenić tytaniczny wysiłek włożony w gcc, g++, czy inne kompilatory (ostatnio wystarczy spojrzeć na repozytorium projektu), by zwątpić, czy językotwórstwo badawcze, eksploracyjne, czy podyktowane zwyczajnie własnymi upodobaniami ma szansę w ogóle się ziścić.


% Każdy z tych etapów stawał się z czasem coraz bardziej złożony:
% gramatyki zwiększały rozmiar, wzrast ze wzrostem poziomu abstrakcji, więc i parsery stawały się coraz bardziej skomplikowane i trudne do napisania, podnosił się też poziom wymagań względem komunikatów o błędach,
% reprezentacje wewnętrznych obiektów musiały stawać się coraz bardziej skomplikowane, zarówno z powodu wzrostu poziomu abstrakcji (potrzeby modelowania np. klas oprócz funkcji, typów generycznych prócz prostych), jak polepszenia optymalizacji, same optymalizacje stały coraz doskonalsze 

%[https://www.researchgate.net/figure/The-large-scale-evolution-of-programming-languages-1953-2012-Lineages-of-influence_fig1_305418676]

https://github.com/stereobooster/programming-languages-genealogical-tree?tab=readme-ov-file
http://rigaux.org/language-study/diagram.html

\section{Jak zmniejszyć koszt wytworzenia kompilatora?}
Jak w wielu dziedzinach informatyki – wykorzystując odpowiednie narzędzia. Kilkadziesiąt lat intensywnego rozwoju teorii kompilacji naturalnie nie przyniosło jedynie rozrostu tematu, lecz i próby jego syntez i automatyzacji podzagadnień w postaci programów i bibliotek. Przytoczmy jeszcze raz diagram faz translacji, zaznaczając na nim tym razem to, co można uprościć z użyciem narzędzi.
\begin{figure}[H]
    \centering
    \includegraphics[height=0.8\linewidth]{images/wstep/fazy_kompilacji.png_bin_antlr_llvm_podpisany.png}
    \caption{Fazy kompilacji, zaczerpnięte ze str. 4 podręcznika\cite{DRAGON_BOOK}, z zaznaczonym pokryciem przez narzędzia}
\end{figure}
Uzyskawszy teorię języków formalnych, gdzie niemal kompletnym opisem składni języka jest jego gramatyka, często w jednolitej notacji Backusa-Naura, nie trzeba było długo czekać, aż podjęto próby automatyzacji procesu tworzenia parserów. Para uniksowych programów z lat siedemdziesiątych – lex i yacc (yet another compiler compiler) ma niebagatelny wpływ na historię informatyki, z ich użyciem stworzono nie tylko tak proste języki jak awk, lecz tak złożone jak perl[ocean of awareness, timeline of parsing], a do 2006 roku[gcc release note] gcc używało ich* do tłumaczenia C.
(*właściwie to ulepszonych ich wersji – flex i bison). 

Programy tego typu przyjmują jako plik wejściowy gramatykę języka w formacie będącym zazwyczaj ich autorskim rozszerzeniem notacji EBNF, zezwalającym często na umieszczanie wewnątrz produkcji gramatycznych bezpośrednio fragmentów kodu i generują parser. Oprócz lexa i yacca obecnie istnieje tak dużo wartych uwagi rozwiązań, że wybór najlepszej opcji może okazać się trudny, zwłaszcza jeśli nie ma się już doświadczenia w temacie. W rozdziale X opisane zostaną nieco dokładniej techniki parsingu i umotywowany zostanie wybór konkretnego generatora parserów. W tym miejscu wystarczy nam wiedza, że dla względnie niewielkich języków, podlegających częstym zmianom i rozwojowi, generowany parser oszczędza wiele pracy – o wiele łatwiej jest zmienić gramatykę języka, niż przepisać (być może znaczną) część procedur ręcznie pisanego parsera rekursywnie schodzącego. Wprawdzie dla wielkich języków programowania parsery ręcznie napisane okazują się mieć w ostatnich latach znaczną popularność, szczególnie dlatego, że można w nich dokonywać wedle woli subtelnych zmian ulepszających komunikaty o błędach, lecz osoba nie będącą ekspertem w dziedzinie… - gdzieś potem, przy parserach
[https://notes.eatonphil.com/parser-generators-vs-handwritten-parsers-survey-2021.html]
[https://www.quora.com/What-is-a-good-parser-generator-for-a-project-Is-YACC-the-best-considering-speed-as-a-factor-too]
[https://lukasatkinson.de/2015/marpa-overview/ - śmieszne]%może do opisu parsingu
[https://rahul.gopinath.org/post/2021/02/06/earley-parsing/]

Znaleźliśmy zatem narzędzia skracające znacząco czas pisania „przedniej” części kompilatora, „środkowa”, jak już powiedzieliśmy bardzo silnie zależy od tłumaczonego języka, więc nie jest przyjaznym środowiskiem dla ogólnych narzędzi, pozostaje jeszcze „tylna” – optymalizacje i generowanie kodu wynikowego. Wracamy w ten sposób do zapowiedzianego LLVM. 

Kompilator clang stworzono, przepisując na nowo gcc, po części z powodów licencyjnych (ze strony Apple pierwotnie finansującego przedsięwzięcie), po części chcąc zastosować nowsze techniki, które nie miałyby miejsca w kodzie programu tak dojrzałego i mającego kluczowe znaczenia dla naszej cywilizacji (nie jest to przesada, skoro kompiluje się przy pomocy gcc w zasadzie wszystko, z jądrem Linuksa włącznie). Oddano przy tym światu dodatkowo wielką przysługę. Warstwowość architektury kompilatora, jak powiedzieliśmy, wynika naturalnie z zasad jego funkcjonowania, a pewna modularność kodu jest koniecznością w tak wielkim projekcie i uznaną obecnie dobrą praktyką programistyczną, w tym jednak wypadku, oprócz wydzielenia back-endu, zdecydowano się wyeksponować go jako osobny projekt - LLVM, definiując publicznie dostępne API, wraz z językiem pośrednim z którego korzysta zarówno kompilator C, jak i C++. Dość szybko zauważono, że można stworzyć własny „front-end” dla dowolnego języka, dającego się tłumaczyć do postaci pośredniej LLVM (LLVM IR), w dodatku otrzymuje się wtedy zupełnie „za darmo” całe nieprzebrane bogactwo optymalizacji potężnych, wymagających zastosowania wysokiej matematyki, bądź zwyczajnie żmudnych w implementacji – wszystkich, które zespół clanga musiał odtworzyć, by choć marzyć o doścignięciu wydajności kodu wynikowego gcc (i późniejszego prześcignięcia). W dodatku zyskujemy możliwość generacji kodu na tyle architektur, na ile portowano clanga, nie jest to liczba tak wielka, jak tych, dla których jakikolwiek kompilator C jest dostępny, jednak i tak przekracza ilość, która uzyskałby nawet duży kompilator pisany od początku do końca. W najbardziej praktycznym sensie, dostajemy za darmo wsparcie dla popularnych architektur, których użytkownikami jesteśmy, albo spotykamy najczęściej się najczęściej – IA-32, (potocznie x68), IA-64 (potocznie x86-64) i rozmiate wersje ARM.

Reprezentacja pośrednia (LLVM IR), a raczej struktury danych i wywołania, które należy użyć, aby zacząć wykorzystywać programowo jej potencjał, nie należą wprawdzie do najbardziej oczywistych, lecz są w zasięgu pojedynczego programisty, co pokazują rozmaite materiały dostępne w sieci[kalleifoscope], jak i zostanie zademonstrowane na praktycznych przykładach w dalszej części pracy.

\section{Dlaczego nie użyć maszyny wirtualnej?}
Nawet po przedstawieniu zalet LLVM, takie pytanie wciąż może niepokoić czytelnika, mając na uwadze, że użycie LLVM wymaga wiedzy niskopoziomowej, podczas gdy pisząc własną maszynę wirtualną, można niemal skutecznie odgrodzić się od sprzętu i skupić się jedynie na bardziej abstrakcyjnych zagadnieniach językowych. 
Odpowiedzią przewrotną na to pytanie jest, że przecież korzystamy przecież z maszyny wirtualnej – LLVM to wszak „Low-Level Virtual Machine”, jedną z jego możliwości jest wygenerowanie statycznego kodu maszynowego, posiada jednak również moduł JIT (just-in-time), reprezentujący tę samą zasadę działania co w roku 2024 cpython. (zob. \_\_pycache\_\_ i .pyc w środku). Żeby odpowiednio rozwikłać te wątpliwość, zbierzmy w tym miejscy wymagania projektowe.
\textbf{Wymagania projektowe względem translatora}
\begin{itemize}[noitemsep]
    \item dla niedużego, łatwo modyfikowalnego (w celu eksperymentów) języka
    \item niewielki, zdatny do napisania przez jedną, dobrze zmotywowaną osobę w kilka miesięcy.
\end{itemize}

Nie jest to długa lista, choć oszacowanie kosztów może wydać się nazbyt optymistyczne, w świetle podanych wcześniej liczb (18 roboczolat dla pierwszego Fortranu), to współczesne narzędzia, jak i ogólny rozwój technik programistycznych, możliwości sprzętowych, czynią redukcję takich rozmiarów na wyciągniecie ręki Nie obieram też za cel stworzenia translatora, jak tamten, zmieniającego postać informatyki, czy „zaledwie” globalnych praktyk programistycznych, ani nawet stworzenia nowego popularnego języka, mam na uwadze jedynie praktyczne i pouczające ćwiczenie w zakresie projektowania języków programowania.

Paradoksalnie ze względu na takie ograniczenia dostępnych zasobów, uważam tworzenie własnej maszyny wirtualnej za gorszą decyzję, o ile tłumaczy się język imperatywny (a wybieram go ze względu na prostotę i niewielką sumę własnych doświadczeń np. z językami funkcyjnmi) Wystarczy uważnie spojrzeć w obecny kod źródłowy cpythona [tu podać plik], w główną pętlę jego maszyny wirtualnej, by pojąć, że napisanie własnej jest również wyzwaniem, wymagającym szczególnej wiedzy, o charakterze być może porównywalnym do tej, koniecznej do pracy z abstrakcjami rzeczywistego sprzętu pokroju LLVM. Najważniejszym czynnikiem dla mnie było wszak oszacowanie kosztu doprowadzenia własnej maszyny wirtualnej do wymaganej niezawodności. Wszelkie pomyłki w niej popełnione wpływałyby na zachowanie całego systemu, zaciemniając problemy w innych częściach translatora i utrudniając ich analizę. 

Używając zarówno generatora parserów, jak i gotowego back-endu, niemal cały wysiłek ściśle programistyczny skupia się w relatywnie niewielkiej środkowej fazie, co zapewnia tak pożądaną redukcję problemu i zlokalizowanie go jednym podsystemie o dobrze określonych granicach. Błędy prozaicznej natury w pozostałych fazach translacji można niemal wykluczyć, ponieważ wykorzystanie narzędzia są intensywnie i profesjonalnie testowane. Występujące problemy można niemal od razu ograniczyć do środkowej fazy, defekty gramatyki objawiają się w odmienny sposób, a niewłaściwe użycie LLVM karane jest zazwyczaj surowo, acz rozpoznawalnie. 

Dodatkowym istotnym czynnikiem, wpływającym na wybór statycznie kompilowanego kodu maszynowego jest kwestia biblioteki standardowej. Praktyka programistyczna sugeruje, że o przydatności języka programowania i subiektywnej ocenie doświadczeń z nim związanych decyduje nie tylko jego składnia, system typów, czy nawet przyjazny edytor (czy dziś raczej IDE), ale w wielkiej części zbiór łatwo dostępnych funkcji (czy procedur) – biblioteka standardowa, ewentualnie możliwość łatwego, powtarzalnego instalowania bibliotek i zarządzania ich zależnościami. Systemy pakietów, choć pojawiają się powszechnie w nowych językach, nawet tak niskopoziomowych jak Rust, wiążą się jednak dodatkowym, znacznym wysiłkiem programistycznym, potrzebnym by w ogóle móc zacząć przydatne pakiety pisać. Nie sądzę, by napisanie praktycznie użytecznej biblioteki standardowej było w zasięgu projektu tego rozmiaru, mając na uwadze, jak wiele pracy włożono w bibliotekę standardową C, ile w C++, a nawet w mnogie praktycznie użyteczne pakiety używane w pythonie, np. do wyrażeń regularnych. Intrygującym pytaniem jawi się systematyczne porównanie ilości pracy włożonej w kompilator danego języka a w jego bibliotekę standardową rozszerzoną o powszechnie używane biblioteki.

Z tego powodu, za najrozsądniejsze rozwiązanie dla prostego, imperatywnego języka uważam skorzystanie z libc, a llvm zapewnia nie tylko taką możliwość, lecz zgodność z binarnym interfejsem C (C ABI) w danym środowisku. Czyli w praktyce wszystko to, co można wywłać bądź użyć z C, jest potencjalnie dostępne również dla języka tłumaczonego z użyciem LLVM.

Pomimo ogromnych możliwości samego llvm w zakresie generacji optymalnego kodu, ze względu na nadrzędne tu dążenie do redukcji zadania, w zakresie efektywności faktycznie generowanego kodu, przyjmijmy jedynie skromne wymaganie: „translator nie produkuje programów rażąco nieefektywnych”. Pozwoli to uchronić się przed wszelkimi zgubnymi formami przedwczesnej optymalizacji i mieć na uwadze charakter języka i docelową grupę jego użytkowników. Kompilatory wielkich języków ogólnego przeznaczenia tłumaczą zarówno kod aplikacji o krytycznym znaczeniu, gdzie ewentualny spadek wydajności w następnej wersji jest automatycznie monitorowany (np. postgres)[postgres regression testing], jak i ogromne wielomegabajtowe źródła  wygenerowane przez transpilery czy domenowe narzędzia, poddające próbie wydajność samego procesu przekładu.

Trzeźwo określmy, że grupą docelową tego projektu będzie co najwyżej wąska grupa osób zainteresowanych, kody źródłowe nie będą wielomegabajtowej wielkości, szybkość działania programów nie będzie najważniejszą metryką. Podwaliny pod zoptymalizowany i szybki kod zapewnia llvm i będzie można je w przyszłości wykorzystać. Możliwości zaś, uwolnione przez to obniżenie wymagań, zostaną spożytkowane w architekturze translatora, prowadząc do jego większej przejrzystości, utrzymywalności i ułatwionego śledzenia błędów.


\section{Podsumowanie wymagań}
Na koniec tego rozdziału zbierzmy w jednym miejscu określone wymagania i cel przedsięwzięcia.

\textbf{Wymagania względem języka:}
\begin{itemize}[noitemsep]
    \item prosty
    \item imperatywny
    \item relatywnie niskopoziomowy
    \item kompilowany
    \item statycznie typowany
\end{itemize}

\textbf{Wymagania projektowe, względem translatora:}
\begin{itemize}[noitemsep]
    \item Niewielki, zdatny do napisania przez jedną, dobrze zmotywowaną osobę w kilka miesięcy.
    \item Modularny, łatwy do debugowania
    \item Łatwy do modyfikacji wraz z rozwojem języka
    \item O niezawodności możliwie zbliżonej do standardu w dziedzinie.
\end{itemize}

\textbf{Wymagania techniczne:}
\begin{itemize}[noitemsep]
    \item Parser generowany wygodnym narzędziem.
    \item Statycznie kompilowany, używając LLVM jako „back-endu”.
\end{itemize}

\textbf{Wymagania pragmatyczne:}
\begin{itemize}[noitemsep]
    \item Wykorzystanie biblioteki standardowej C, potencjalnie innych, dostępnych przez C ABI.
    \item Translator nie produkuje programów „rażąco niefektywnych”
\end{itemize}
	\chapter{Projekt języka}
\label{cha:wywodJezyka}
%jakieś podnagłówki
W literaturze definiuje się język jako podzbiór zbioru skończonych ciągów symboli nad pewnym alfabetem”\cite{hopcroft_automaty}. Definicją taka zachęca, aby dokonać z pozoru arbitralnego wyboru elementu tej przestrzeni podzbiorów i bez wyjaśnień opisać go, najlepiej w formie gramatyki bezkontekstowej, która to forma stała się praktycznym standardem w konstrukcji kompilatorów. Świadczy o tym definicja użyta w jednym z najważniejszych artykułów w dziedzinie translacji – „Parsowaniu języków od lewej, do prawej” Donalda Knutha z 1966\cite{TRANSLATION_FROM_LEFT_TO_RIGHT}, opisującej mechanizm, który dzisiaj nazywa się parserem LR. 
„The language defined by G is
\[
\{ \alpha \mid S \Rightarrow \alpha \text{ and } \alpha \text{ is a string over } T \}
\]
namely, the set of all terminal strings derivable from S by using the productions of G as substitution rules.”

W wolnym tłumaczeniu:
„język zdefiniowany przez gramatykę G to \[
\{ \alpha \mid S \Rightarrow \alpha \text{ i } \alpha \text{ to napis nad } T \}
\]czyli zbiór wszystkich ciągów terminali wyprowadzalnych z symbolu startowego S poprzez zastosowanie G jako zbioru produkcji”.

Powyższą definicja jest kontynuacją myśli wyrażonej jeszcze w 1926 roku przez Bloomfielda\cite{BLOOMFIELD_1926}, który poszukując pierwszej w zachodniej nauce formalizacji lingwistyki, chcąc wyzbyć się wszelkich odniesień do „mentalnych stanów”, uznawanych przez ówczesne poglądy za niemierzalne, a przez to nienadające się do budowy empirycznej nauki, skupił się jedynie na zewnętrznych, a przez to opisywalnych przejawach języka, a przez to jego zewnętrznej reprezentacji,  czyli ekstensji języka.\cite{parsing_timeline_kegler}

Wszelako poznanie „intensji” języka, czyli dobre zrozumienie koncepcji stojących za nim, zdaje się niezwykle pożyteczne, gdy konstruuje się jego translator. Istnieje pogląd, który Peter Naur, (przytoczony w tym miejscu bo właśnie on jest twórca powszechnej notacji gramatyk formalnych), wyraża w swym artykule „Programowanie jako budowanie teorii” – jakoby kod stanowił zaledwie mniejszą część programu, a większą i istotniejszą jest mentalny model w umyśle programisty go piszącego\cite{NAUR_1985}. Dlatego też spróbujemy wpierw osiągnąć pewne wyobrażenie o projektowanym języku, a dopiero potem ująć go formalnie. 

Sam proces projektowania nowych języków jawi się dość tajemniczo, ponieważ twórcy języków rzadko opisują go bezpośrednio. Czytamy jedynie o wnioskach wyciągniętych post factum – np. Dennis Ritchie stwierdził, że należało przyjąć inną precedencję operatorów || i \&\&\cite{Ritchie_mail}, czy w najlepszym wypadku stwierdzenia w rodzaju wygłoszonego przez Wirtha, że twórca języka powinien być ,,rozsądnym zbieraczem rozwiązań'' (judicious collector of features)\cite{Wirth_recollections_Pascal}. Z tego, co możemy dowiedzieć się z literatury, można złożyć obraz dwóch przeciwstawnych sobie kierunków, sił rywalizujących ze sobą przy projektowaniu języka:

Pierwszą jest pewien rodzaj pryncypium najmniejszych zaskoczeń, zwłaszcza w dzisiejszych czaszach, gdy języki wysokiego poziomu stały się powszechne i pewien ich kształt jest oczekiwany przez programistę, stanowiąc niemal lingua franca, i umożliwiając pobieżne czytanie nieznanych wcześniej acz spokrewnionych języków. Ująć to możną również  że każdy język wysokiego poziomu, a w szczególności imperatywny, zawrzeć musi w sobie interpretację
\begin{enumerate}[noitemsep, label=(\alph*)]
    \item Wyrażeń matematycznych
    \item Podstawowych struktur algorytmicznych
\end{enumerate}
na swój sposób musi zarazem być FORTRANEM – translatorem formuł matematycznych oraz ALGOLEM – językiem opisu algorytmów.
Wypełnienie tych dwóch oczekiwań wprowadza ustalone i dobrze poznane struktury gramatyczne do języka.

Przeciwstawna do chęci zaspokojenia oczekiwań ludzkiego umysłu, jest konieczność dostosowania się do technicznych możliwości parsera. Często to, co dla człowieka jest naturalne i czytelne, jest praktycznie niemożliwe do automatycznej analizy składniowej.  Sama teoria parsingu i jej dokładne ograniczenia jest zbyt złożona, aby ją przedstawić w takiej pracy, choć przegląd metod pojawi się przy opisie wybranego generatora parserów. Na szczęście, dwa wymienione metajęzyki (wyrażeń i prostych struktur algorytmicznych), doczekały się niezliczonych implementacji i gramatyki je opisujące są dobrze poznane. Zacznijmy zatem prostym przykładem
\begin{lstlisting}
dopóki(h>0)
{
    Dv = g * Dt - opór(v, h);
    v += Dv;
    jeśli( abs(Dv) < eps_Dv) {prędkość_graniczna = v;}
    h -= v * Dt;
}
jeśli(prędkość_graniczna == 0)
    {pisz("Nie wyznaczono prędkości granicznej");}
inaczej {pisz("Przybliżeniem prędkości granicznej jest %f, jeśli funkcja oporu aerodynamicznego ma pewne własności.");}
\end{lstlisting}


Obrazuje to chyba dość dobrze oczekiwania wobec dowolnego języka imperatywnego – niezależnie od szczegółów, proste pętle, warunki i wyrażenia matematyczne (lub fizyczne) zawsze będą wyglądać podobnie, nawet jeśli wychodzilibyśmy od składni języka innego niż C. (A przynajmniej będą dostępne łatwo rozpoznawalne wersje podstawowych struktur kontroli przepływu - wiele jest form i semantyk instrukcji \textit{for} w różnych językach, lecz po \textit{while} można się spodziewać niemal zawsze tego samego.)

O ile zapis wywołania funkcji ma postać ustaloną powszechnym oczekiwaniem (oraz matematyką), o tyle więcej możliwości pojawia się przy jej deklaracji.
\begin{lstlisting}
rzeczywista opór(rzeczywista v, h)
{
	zwróć -6*PI*v^2*M*R;
}
\end{lstlisting}

\begin{lstlisting}
fun  opór(rzeczywista v, h) -> rzeczywista
{
	zwróć -6*PI*v^2*M*R; //formuła Stokesa
}
\end{lstlisting}
\begin{lstlisting}
opór = lambda v,h: -6*PI*v^2*M*R;\end{lstlisting} - jeśli gotowi jesteśmy na implementację inferencji typów.
albo nawet 
\lstset{
    escapechar=|,
    breaklines=true
}
\begin{lstlisting}
opór = |$\lambda$| v.h. -6*PI*v^2*M*R\end{lstlisting} – jeśli chcemy zmuszać użytkownika do zaopatrzenia się w grecką klawiaturę i złamać oczekiwania w kwestii semantyki kropki
\lstset{
    escapechar=@
    breaklines=true
}
\begin{lstlisting}
def f(x: int) -> int:               	 Python
function f(x: number): number { ... }    TypeScript
fun f(x: Int): Int { ... }          	 Kotlin
int f(int x) { ... }                	 C
auto f(int x) -> int { ... }        	 C++ (nowsza składnia)
int f(x int) { ... }                	 Go
f(x: int) -> int do ... end         	 Elixir
f(x :: Int) :: Int = ...            	 Haskell
function f(x) return x + 1 end      	 Lua
def f(x: Int): Int = ...            	 Scala
func f(x int) int { ... }           	 Go
public int f(int x) { ... }         	 Java
fun f(x: Int): Int = x + 1          	 Kotlin
let f x : int -> int = x + 1        	 OCaml
fn f(x: i32) -> i32 { ... }         	 Rust
Function f(x As Integer) As Integer 	 VB.NET
\end{lstlisting}
\lstset{
    escapechar=|,
    breaklines=true
}
Jeśli dobrze przyjrzymy się dostępnym opcjom, to spostrzeżemy, że wszystkie chyba możliwości zostały przetestowane w jakimś powszechnie dostępnym języku. Możemy umieszczać typ przed, jak i po nazwie parametru, używać różnych odmian słowa kluczowego fun/function, typu zwracane umiejscowić przed nazwą funkcji, lub po, parametry przekazywać przez wartość, referencję, nazwę et cetera - dla każdej rozsądnej kombinacji konstrukcji składniowych prawdopodobnie można by znaleźć język jej używający.
Czy zatem nic ciekawego nie pozostało do zrobienia z funkcjami, mimo że mamy niepowtarzalną okazję stworzenia własnej gramatyki? (jęz.)

\section{Wielokrotne punkty wejściowe}
Natrafiłem kiedyś na podręcznik do dawno wymarłego już języka PL/I i zacząłem przeglądać z pewnym zainteresowaniem, ponieważ jest on odległym przodkiem języka procedur bazodanowych w Postgresie. Zwróciła moją uwagę strona 196 z podręcznika\cite{plif} oraz następna.
\begin{figure}[h]
    \centering
    \includegraphics[width=1.3\textwidth]{images/wywod/PLI1_bin.png}
    \caption{Widok dekompilacji w programie GHIDRA}
\end{figure}
%[]
Przyjrzawszy się przykładowi, możemy skonstatować, że znamy obecnie znacznie poręczniejszą składnię argumentów domyślnych i należy pozostawić szybko za sobą ten drobny przypis do składni wymarłego języka. Ja wszak nie potrafiłem do końca zapomnieć o tej dziwacznej konstrukcji, nosząc przeświadczenie, że widziałem ją już wcześniej w innym miejscu. Systematyczna kwerenda ujawnia, że podobna składnia ze słowem kluczowym ENTRY pojawia się również w fortranie[strona z Bieleckiego – 120], oraz w COBOLU. Pewien czas wszakże upłynął, zanim znów przypadkowo dotarłem w miejsce, gdzie owo „entry” zobaczyłem po raz pierwszy -  na liście słów kluczowych C u Kernighana i Ritchiego, opatrzone wyjątkowo lakonicznym opisem, nie ujawniającym powiązanej z nim składni [K\&R, A/200]
W wydaniu drugim tej książki pojawia się: [K\&R 2, …?]
odpowiadając już powszechnemu doświadczeniu C, jako języka pozbawionego takiej konstrukcji.
Szerzej zakrojone wysiłki w zakresie archeologii językowej, przedstawiają „ENTRY” i formy podobne, jako dziwaczne skamieliny – występujące powszechnie w pierwszym pokoleniu języków wysokiego poziomu, by potem stopniowo wymrzeć, poprzez zepchnięcie na marginesy standardów, zarzucenie, jak w C, czy zupełne zapomnienie, nie tylko przez programistów, lecz i projektantów języków w czasach późniejszych. W standardach i podręcznikach odbywa się to wymieranie bez obszerniejszych wyjaśnień, a w różnych mniej formalnych tekstach i urywkach, powtarzane są często opinie, jakoby konstrukcja ta miała być (jeszcze w latach 70) przestarzała, niezalecana,  passe, albo wręcz szkodliwa, otoczona aurą, charakteryzującą zazwyczaj rozwiązania wykorzystywane nagminnie w złych praktykach programistycznych. Przytoczę jeden przykład z forum (w odniesieniu do C):
„Really, it isn't a big loss. The concept of one subroutine with
multiple entry points was on its way out even before C was new, and it
was wise of the standards makers to ignore such a half-baked idea."
*[tłum]
„Naprawdę, to nie jest wielka strata. Koncepcja jednej procedury z wieloma punktami wejścia była w odwrocie jeszcze zanim C stał się nowy, i było mądre ze strony twórców standardów, że zignorowali tak niedopracowany pomysł.” 
 
Nawet jeśli komentarze tego typu wygłaszane są przez programistów dość sędziwych, by pamiętać rzeczywiste uzasadnienie rezygnacji z pomysłu, uznają to za fakt dawno dokonany i uzasadniony, niestety nie dzieląc się owymi wyjaśnieniami na piśmie, przynajmniej nie dość często, by autor tej pracy je odnalazł.
Nawet bez uzasadnień z epoki, można łatwo stwierdzić, że owa konstrukcja składniowa jest prostą konsekwencją przeniesienia sposobu pisania procedur wprost z języków niższego poziomu. Wyobraźmy sobie, że programista asemblerowy, w latach 60 napisał procedurę do drukowania zawartości tabeli. Krótka funkcja w C:

\begin{lstlisting}
void write_table(char *** cells, char ** colnames, int num_cols, int num_rows)
{
    for(int i=0; i<num_rows; i++)
    {
        if(i>0){putchar(',');}
        printf("%s", colnames[i]);
    }
    printf("\n")
    for(int j=0; j<num_rows; j++)
    {
        for(int i=0; i<num_rows; i++)
        {
            if(i>0){putchar(',');}
            printf("%s", cells[j][i]);
        }
        printf("\n");
    }
}
\end{lstlisting}

Przed językami wysokiego poziomu, byłaby obszerniejsza i mogłaby wyglądać tak:
(w celach demonstracyjnych użyto współczesnego asemblera NASM na architekture IA-32 (x86))

\begin{lstlisting}
write_table:
; Write column names
    xor eax, eax       ; Reset eax for loop variable
    mov ecx, num_cols  ; Number of column names to write
write_colnames:
    cmp eax, ecx       ; Compare current index with num_cols
    jge write_cells    ; If done, jump to write cells

    mov ebx, [colnames + eax * 4] ; Get pointer to current colname
    test eax, eax                 ; If index > 0
    jz .write_name                ; Skip comma on first colname
    mov esi, comma
    call write_string

.write_name:
    mov esi, ebx
    call write_string

    inc eax          ; Increment column index
    jmp write_colnames

write_cells:
    ; Loop through rows
    xor eax, eax      ; Row index
    mov edx, num_rows ; Total rows
row_loop:
    cmp eax, edx
    jge done          ; Exit after all rows are written
    push eax          ; Save row index

    ; Loop through cells in a row
    xor ecx, ecx      ; Reset column index
    mov esi, [cells + eax * 4] ; Get pointer to current row
write_row:
    cmp ecx, num_cols
    jge .next_row     ; Break when all cells in the row are written

    mov edi, [esi + ecx * 4] ; Pointer to cell
    test ecx, ecx
    jz .write_cell    ; Skip comma before first cell
    mov esi, comma
    call write_string

.write_cell:
    mov esi, edi
    call write_string

    inc ecx          ; Increment column index
    jmp write_row

.next_row:
    call write_newline
    pop eax          ; Restore row index
    inc eax
    jmp row_loop
\end{lstlisting}

Wyobraźmy sobie, że ta procedura ta używana jest wielokrotnie w istniejącym oprogramowaniu:
\begin{lstlisting}
...zapisywanie niektórych rejestrów (caller-save registers)
... ustawianie rejerestrów - argumentów
call write_table
...przywracanie rejestrów zapisywanych przez wołającego
\end{lstlisting}
Przypuśćmy, że pewnego dnia księgowy przychodzi do programisty utrzymującego to oprogramowanie i prosi, aby w dwóch z szesnastu przypadków użycia tej procedury, jednak nie drukować nagłówków. Dzisiaj lista parametrów funkcji write\_table zwyczajnie powiększyłaby się o flagę, określającą, czy pisać nagłówek. W asemblerze również można to zwięźle zapisać – dodać na początku procedury
\begin{lstlisting}
jz %rejestr_z_flagą write_cells
\end{lstlisting}
Trzeba jednak wtedy zmienić każde z wywołań procedury, jest to bardziej pracochłonne w języku niskiego poziomu, a jeśli argumenty są przekazywane w rejestrach, nie na stosie (co zdarza się w niektórych architekturach i dzisiaj), to konieczność znalezienia dodatkowego wolnego rejestru miejscu wywołania może zaburzyć ręcznie skonstrowany schemat przydziału rejestrów w okalającej procedurze. (A dowiedziono że problem przydziału rejestrów jest NP-zupełny.[]) Dlatego dość prawdopodobne, że programista zwyczajnie w dwóch miejscach, zamiast
\begin{lstlisting}
call write_table
\end{lstlisting}
napisałby 
\begin{lstlisting}
call write_cells
\end{lstlisting}
write\_table i write\_cells są w istocie jedynie adresami i gdy nie ma rozbudowanego prologu procedury, można ją rozpocząć równie dobrze w innym punkcie w ten sposób. Możemy więc wyobrazić sobie, dlaczego programistom przywykłym do języków niskiego poziomu, taka konstrukcja wydałaby się naturalna:
\lstset{
    escapechar=|,
    breaklines=true
}
\begin{lstlisting}
void write_table(char *** cells, char ** colnames, int num_cols, int num_rows)
{
    for(int i=0; i<num_rows; i++)
    {
        if(i>0){putchar(',');}
        printf("%s", colnames[i]);
    }
    
    |\textbf{entry}| write_cells(char *** cells, int num_cols, int num_rows);
    for(int j=0; j<num_rows; j++)
    {
        for(int i=0; i<num_rows; i++)
        {
            if(i>0){putchar(',');}
            printf("%s", cells[j][i]);
        }
    }
}
\end{lstlisting}

Nie dziwi też, że takie relikty mogły być zwalczane, gdy próbowano wypracować i rozpowszechnić wśród praktyków, oczywiste dziś pojęcie funkcji, jako bloku kodu robiącego jedną rzecz, z paramtrami wejściowymi oraz wartością zwracaną[K\&R??]. Inną konstrukcję bezpośrednio pochodzącą z asemblerów – goto rugowano bardzo długo, ochoczo i niemal z całkowitym sukcesem, podobnie jawne etykiety do niej potrzebne, składniowo skopiowane bezpośrednio z asemblerów. Czy może mieć dla nas ów zabytek paleografii jakiekolwiek znaczenie? Zauważmy, że opisuje on ciekawy byt: funkcja ma z algorytmicznego punktu widzenia jedno wejście i wiele wyjść, on ma zarówno wiele wyjść, jak i wejść.
[rysunki dwa funkcji i proc wielow. Jako pudełek ze strzałkami]
Wszystkie historyczne implementacje wyróżniają jeden – główny oraz zero lub więcej punktów wejściowych, co zawsze wygląda dość podobnie:
\lstset{
    escapechar=|,
    breaklines=true
}
\begin{lstlisting}
|\textbf{pisz\_tabelę}|(znak ^^^ krotki,znak ^^ nazwy_kolumn, całk liczba_kolumn, całk liczba_wierszy)
{
    dla(całk i=0; i<liczba_wierszy; i++)
    {
        jeśli(i>0){putchar(',');}
        printf("%s", colnames[i]);
    }
    
    |\textbf{wejście}| pisz_krotki(znak ^^^ krotki, całk liczba_kolumn, całk liczba_wierszy);
    dla(całk j=0; j<liczba_wierszy; j++)
    {
        dla(całk i=0; i<liczba_wierszy s; i++)
        {
            jeśli(i>0){putchar(',');}
            printf("%s", cells[j][i]);
        }
    }
}
\end{lstlisting}
W ramach eksperymentu wznieśmy hasło: „Równe prawa dla wszystkich punktów wejściowych!”, przesuńmy nawiasy,  zamieńmy słowo kluczowe „wejście” na krótszy odpowiednik i otoczmy całą procedurę odpowiednim nagłówkiem:
\begin{lstlisting}
procedura {
|\textbf{tu}| pisz_tabelę(znak ^^^ krotki,znak ^^ nazwy_kolumn, całk liczba_kolumn, całk liczba_wierszy) -> Nic;
    dla(całk i=0; i<liczba_wierszy; i++)
    {
        jeśli(i>0){putchar(',');}
        printf("%s", colnames[i]);
    }
    
|\textbf{tu}| pisz_krotki(znak ^^^ krotki, całk liczba_kolumn, całk liczba_wierszy)->Nic;
    dla(całk j=0; j<liczba_wierszy; j++)
    {
        dla(całk i=0; i<liczba_wierszy s; i++)
        {
            jeśli(i>0){putchar(',');}
            printf("%s", cells[j][i]);
        }
    }
}
\end{lstlisting}
Uzyskaliśmy w ten sposób dość oryginalną składnię, która nie jest już reliktem asemblerów, lecz organiczną częścią języka*. Teraz będzie można eksperymentalnie sprawdzić, czy wielowejściowe procedury są rzeczywiście szkodliwe, lub kłopotliwe w praktyce, czy też da się je kreatywnie wykorzystać. Można rówież powiedzieć, że przypominamy w ten sposób programiście, że nie pracuje z matematycznymi funkcjami, które udają dobrze języki funkcyjne, lecz w prostym, imperatywnym i przede wszystkim proceduralnym języku, nie mającym wielkich ambicji względem abstrakcji matematycznych.
*Implementacja w gramatyce bezkontekstowej jest prosta: nieterminal sygnatury funkcji  poprzedzamy słowem kluczowym i traktujemy jako jedną z instrukcji, na równi z np. warunkiem, czy pętlą, zob instrukcja\_wkroczenia w gramatyce w dalszej części.

\section{System typów}
W poprzednich przykładach niepostrzeżenie pojawiły się nazwy typów. „Typ określa możliwe operacje na bitach oraz uzgodnienia [konwersje] jakie wolno na nich stosować” [Konstrukcja kompilatorów s. 259.]. Język wysokiego poziomu nie może obejść się bez spójnego systemu typów, gdyż sprawdzanie ich jest jedną z głównych aspektów analizy semantycznej programu, typy niekiedy sterują też procesem przekładu.
Na system typów składa się zbiór typów podstawowych (atomicznych), zbiór konstruktorów typów, czyli reguł łączenia ich w typy złożone (np. struktur, tablic, czy funkcji – typy „strzałkowe”), oraz określenie operacji na nich dozwolonych, w tym specjalnych unarnych operacji do konwersji pomiędzy typami.
\begin{lstlisting}
całkowite x = 0;
x = y + 7.0;
\end{lstlisting}
Mając pewne użycie operacji f, (funkcji, lub operatora), oraz n-elementową krotkę użytych argumentów: (t1, t2, … tn), trzeba dysponować mechanizmem określającym, czy dane typy są dozwolone dla tej operacji. W bardziej skomplikowanych przypadkach, gdy istnieje wiele operacji o tej samej nazwie,  stanowiących przeładowania nazwy lub operatora, mechanizm translatora musi wybrać właściwe dopasowanie argumentów do parametrów formalnych, spróbować dobrać odpowiednie konwersje automatyczne, lub zgłosić zrozumiały błąd.
W powyższym przykładzie, programista spodziewa się, że druga linia zostanie przetłumaczona bez zastrzeżeń, gdyż przywykł do oglądania takich napisów w kontekście matematycznym (fizycznym). Wymaga to wszak zastosowania różnych automatycznych konwersji, w zależności od typu zmiennej y.
\begin{lstlisting}
całkowite y; => x = całkowita(rzeczywista(y) + 7.0); lub x = y + całkowita(7.0)
rzeczywiste y; => x = całkowita(y + 7.0)
znak y; => //podobnie jak całkowita, lub błąd, zależnie od reguł konkretnego języka, np. w Pythonie trzeba jawnie napisać 
Typ strukturalny T => błąd, chyba, ze zdefiniowano przeładowanie operatora + dla T, całk.
\end{lstlisting}
Zagadnienie dobrego zaprojektowania systemu typów, przestaje być proste, gdy włączymy do niego wszystkie powszechnie używane mechanizmy – konwersje jawne i automatyczne, przeładowania nazw i operatorów, polimorfizm oraz typy parametryzowane. W szczególności te ostatnie wydają się tak skomplikowanym konceptem, że często okazują się wykraczać poza wyobrażenia ich twórców\cite{cpp_templates_turing_complete}\cite{taming_wildcards_java}, z pewnością zaś przekraczają możliwości tak małego projektu.
W warunkach tak ograniczonych, na początku rozwoju języka, właściwym pytaniem wydaje się być „Jaki jest najprostszy możliwy w tym wypadku system typów?” Odpowiedź zapewnia system typów maszyny docelowej - LLVMa, będący z kolei pierwotnie przystosowany do języków C i C++. Skopiujemy więc w zasadzie system typów C, próbując go wszakże możliwie uprościć.
Typy podstawowe podyktowane są nawet nie przez samo LLVM, lecz przez architektury sprzętu, dysponującego dziś rejestrami całkowitymi o długościach od bajtu do słowa maszynowego (32/64 bity) oraz rozmaitymi rejestrami zmiennoprzecinkowymi, przechowującymi 32 lub 64 bitowe  reprezentacje liczb zmiennoprzecinkowych podług standardu IEEE754. (Pomijamy rejestry wektorowe)
\begin{center}
\begin{tabular}{|c|c|c|}
\hline
\textbf{Projektowany język} & \textbf{LLVM} & \textbf{C/C++} \\ \hline
znak                        & i8       & char               \\ \hline
całkowita/całk              & i64      & long long int      \\ \hline
rzeczywista/rzecz           & f64      & double             \\ \hline
\end{tabular}
\end{center}

Ascetycznie opuszczamy na razie liczby bez znaku, oraz mając okazje tworzyć język od początku, bez wymogów kompatybilności wstecznej, ignorujemy kłopotliwą i niewiele wnoszącą w praktyce dychotomię liczb 32/64bitowych. (LLVM przetłumaczy poprawnie arytmetykę na liczbach 64bitowych również na rozkazy maszyny 32 bitowej, zajmować wszak będzie każda zmienna dwa rejestry, nie będzie to kod efektywny. W roku pisania tej pracy maszyny 32 nitowe są już wyraźną mniejszością)
Nie są to jednak wszystkie typy potrzebne do funkcjonalnego systemu. Konieczne są jeszcze tablice:
\begin{lstlisting}
rzecz^ tabl = nowe [17] rzecz;
\end{lstlisting}
Jako że LLVM posiada wbudowane malloc, kopiujemy składnię alokacji na stercie z C++, przesuwając jedynie liczbę żądanych elementów przed typ.
Do oznaczania wskaźników nie trzeba używać *. Zapożyczamy semantykę znaku \^ z niestandardowej implementacji C++ dla .NET (Microsoft C++ CLI). Tam oznaczała ona uchwyt, lub inaczej referencję zarządzaną przez odśmiecacz.(gdyż to .NĘT i ze zwykłym C++ miał niewiele wspólnego) W projektowanym języku oznaczać będzie jednak \^ zwykły (nagi wskaźnik), jak w C, lecz należy zaznaczyć, że LLVM oferuje szereg mechanizmów integracyjnych dla odśmiecaczy (garbage collectors) i w dalszym etapie rozwoju projektu można rozważyć użycie, np. odśmiecacza Boehma, jak czyni to język V.\cite{vlang_repo}
Zmiana oznaczenia krotności wskaźnikowej ma jednak przede wszystkim na celu zaznaczenie odmiennej od C semantyki dla prostych referencji do struktur. W C/C++ typy złożone używa się zarówno bezpośrednio (np. zajmując pamięć na stosie)
\begin{lstlisting}
typedef struct A {int b, char c;}A;
<wewnątrz funkcji>
A a;
a.b + (a.c- 'a');
\end{lstlisting}
Lub używa wskaźnika do nich (a w C++ również „referencji”)
\begin{lstlisting}
A* a = new A; albo A* a = malloc(sizeof(A));
a->b + (a->c - 'a');
\end{lstlisting}
Konieczność umieszczenia bardzo podobnego kodu w translatorze dla semantyki struktur „bezpośrednich” i operatora „.” oraz wskaźników do struktur z operatorem →, wydaje się w pierwszym podejściu niezachęcające. Obserwujemy też, że wiele języków znakomicie funkcjonuje, używając wyłącznie wskaźników do obiektów, nazywając je referencjami i rezygnując z arytmetyki na nich, np. Java
\begin{lstlisting}
A a = new A(); 
a.b + (a.c- 'a');
\end{lstlisting}
czy Python
\begin{lstlisting}
a = A() 
a.b + (ord(a.c)- ord('a'));
\end{lstlisting}
W odwrotnym kierunku podążył język Go. W nim wskaźniki są konsekwentnie oznaczane przez gwiazdkę, a jeśli jej nie ma, przekazuje się obiekt w całości przez wartość, z wszystkimi tego konsekwencjami.
W żadnym wymienionych języków oprócz C/C== nie zachowano dychotomii ./→.
Mając na uwadze prostotę translatora, zawsze przekazujmy obiekty przez referencje, używajmy kropki przy dostępie do składowej, a znak \^, niech oznacza „podniesienie krotności wskaźnikowej” w następujący sposób:

\begin{center}
\begin{tabular}{|c|c|c|c|}
\hline
\textbf{Projektowany język} & \textbf{C} & \textbf{LLVM \tiny{(opaque pointers)}} & \textbf{,,Krotność wskaźnikowa''} \\ \hline
całk        & long long int             & i64 & 0 \\ \hline
całk\up     & long long int *           & ptr & 1 \\ \hline
całk\up     & long long int <nazwa>[]   & ptr & 1 \\ \hline
całk\up\up  & long long int **          & ptr & 2 \\ \hline
-           & T                         & T   & 0 \\ \hline
T           & T*                        & ptr & 1 \\ \hline
-           & T <nazwa> []              & ptr & 1 \\ \hline
T\up        & T**                       & ptr & 2 \\ \hline
T\up\up     & T***                      & ptr & 3 \\ \hline
\end{tabular}
\end{center}

Widzimy też, że można wyzbyć się chwilowo tablic o znanej wielkości, używając w ich miejsce wskaźników. W C int *t  nie jest równoważne, int t[] a już na pewno nie int t[20],w sensie równości typów, dwa ostatnie zachowują się jednak, jakby były podtypem pierwszego (zwie się to rozpadem tablic do wskaźników - „pointer decay”). Zważywszy na to, że tak jak C nie jesteśmy zapewnić zadowalającego dla dzisiejszych zastosowań sprawdzania granic tablic, uproszczenie pozwalające indeksować dowolny wskaźnik wydaje się być akceptowalne – do tablic wystarczą typy wskaźnikowe.
\begin{lstlisting}
całk ^ t; całk i = t[3];
\end{lstlisting}
Struktury niech otrzymają możliwie prostą składnię:
typ Zespolona {rzecz rz; rzecz ur;};
Definiujmy w ten sposób jedynie tablice wskaźników,  lecz nie samych obiektów:
\begin{center}
\begin{tabular}{|c|c|}
\hline
\textbf{Projektowany język} &  \textbf{C} \\ \hline
Zesp z; z[1] - niedozwolone & Zesp * z; z[1] – dozwolone – tablica struktur \\ \hline
Zesp\up \space z; z[1] – dozwolone – tablica wskaźników & Zesp * z[]; z[1] – dozwolone – tablica wskaźników \\ \hline
\end{tabular}
\end{center}

Takie poświęcenie oznacza kilka procedur w translatorze mniej… nie wydaje się również, żeby programista, który już zacznie myśleć w kategoriach bardziej współczesnych referencji (takich jak w Javie), przypominał sobie zbyt często o tablicach struktur rozmieszonych jednorodnie w pamięci.

Pozostaje jeszcze do rozstrzygnięcia kwestia pustych referencji. Powszechnie wskazuje się na nie, jako na przyczynę różnych problemów, błędów i podatności. Niestety do ich skutecznego, lecz jednocześnie znośnego dla programisty wyeliminowania, potrzebny jest o wiele potężniejszy system typów i mechanizm zarządzania pamięcią – garbage collector lub borrow checker. Zmuszeni jesteśmy zatem zaakceptować napisy podobne do poniższego:
\begin{lstlisting}
T t = pusty;    
\end{lstlisting}
 gdzie „pusty” odpowiada wyróżnionej wartości oznaczającej nie wskazywanie na obiekt. Powstaje  pytanie, jakiego typu powinien być ten specjalny literał. Dość powszechnie wiadomo, że w języku C pierwotnie makro NULL rozwijane było do 0.\cite{KiR} To dość naturalne, w kontekście wczesnego C, jeszcze przed standardem ANSI. System typów był wtedy mniej ścisły, w wielu miejscach można było je opuszczać, a kompilator domyślnie zakładał typ całkowity, w dodatku używając go automatycznie również jako typu wskaźnikowego. Krytykę tak swobodnego podejścia, rozmontowującego w praktyce kontrolę typów można znaleźć w pierwszym wydaniu podręcznika Kernighana i Ritchiego(\cite{KiR}). Gdy jednak wprowadzono  typy void * do C, rozsądniejszą wartością dla NULL stało się ((void*)0), co przynajmniej wyeliminowało niejawne rzutowania z typu całkowitego na wskaźnikowe, lecz dalej nie jest to stan satysfakcjonujący:
\begin{lstlisting}
(wyrażenie) T* t = NULL;
(typy)         T*  void *
\end{lstlisting}
T* jest podtypem void*, lecz nie na odwrotnie, ściśle rzecz ujmując jest to podstawienie niedozwolone, tak jakby w Javie napisać np.
\begin{lstlisting}
OutputStream s = (Object)obj;
\end{lstlisting}
lub ogólniej:
\begin{lstlisting}
class A extends B[...] B b; A a = b;
\end{lstlisting}
Nie można zezwalać na niejawną konwersję z void* do dowolnego typu, niweczy to efektywność statycznego typowania, które ma zagwarantować, że nie da się wykonać operacji na obiekcie niewłaściwego typu. Rozwiązaniem stosowanym w C++, choć znanym już dużo wcześniej [Konstrukcja Kompilatorów str 256 – nil\_typ], jest wprowadzenie specjalnego, osobnego typu, którego jedynym okazem jest literał pustej referencji, a następnie zdefiniowanie domyślnej konwersji od niego do dowolnego konkretnego typu referencyjnego (lecz nie do void*). W ten sposób możemy zachować jedynie jawne konwersje do i z nieokreślonego typu referencyjnego,  zachowując funkcjonalność pustego literału. W C++ ten typ nazwano nulptr\_t*, w programie rozwijanego translatora nosi nazwę TpPustego i podobnie jak w C++, użytkownik nie musi być świadom jego istnienia.
(*Ze względu na kompatybilność wsteczną, nie zmieniano definicji makra NULL, lecz dodano nowy literał nullptr, mający go zastępować)
\begin{lstlisting}
T t = pusty;
T = TpPustego^ - typy - przerobić na obrazek
T * t = nulptr;
T* = nulptr_t
\end{lstlisting}
Możemy więc podać cały diagram konwersji w tworzonym języku:


\begin{figure}[h]
    \centering
    \includesvg[width=0.6\textwidth]{images/3.konwersje.svg}
    \caption{Diagram konwersji w projektowanym języku}
\end{figure}

Przyjąłem zasadę, żeby nie definiować konwersji automatycznych, które są jednocześnie zwężające (prowadzą do utraty informacji)

\section{Gramatyka języka}
Język programowania można dość dokładnie opisać w formie dobrze znanej w matematyce definicji rekurencyjnej, jak czyni to wstępny opis ALGOLA\cite{ALGOL_PRELIMINARY_REPORT} czy SAKO\cite{SAKO}. Aby skonstruować parser dla danego języka, potrzebny jest wszak ścisły opis i w roli tej znakomicie sprawdza się gramatyka bezkontekstowa.

Wyjaśniwszy co istotniejsze decyzje podjęte przy projektowaniu języka, można w tym miejscu zamieścić jego gramatykę w całej rozciągłości.
Zostanie ona przedstawiona w postaci nieznacznie różniącej się od klasycznej rozszerzonej notacji Backusa-Naura (EBNF), a będącej rzeczywistym plikiem źródłowym dla generatora parserów wykorzystywanym w projekcie. Ma to na celu zademonstrowanie użyteczności takich gramatyk nie tylko, jako specyfikacji konstrukcji parsera, lecz jako zwartej, czytelnej dla człowieka i przede wszystkim ścisłej definicji składni języka.

\newgeometry{left=1cm,right=1cm,top=2cm,bottom=2cm} % Set thinner margins
\lstset{
    escapechar=@,
    breaklines=true
}
\begin{lstlisting}[basicstyle=\scriptsize\ttfamily,breaklines=true]
grammar plpl;
program : (byt_globalny)* EOF;
byt_globalny: procedura | deklaracja_typu | deklaracja_nazwy;
deklaracja_typu: deklaracja_aliasu_typu | deklaracja_typu_strukturalnego;
deklaracja_aliasu_typu:'typ'  nowa_nazwa_typu 'jak' NAZWA_TYPU ';';
deklaracja_typu_strukturalnego   : 'typ'  nowa_nazwa_typu  '{' ( deklaracja_nazwy )* '}' ';';
nowa_nazwa_typu:  NAZWA | NAZWA_TYPU; //użytkownik wprowadza nowy typ


procedura   :  PROCEDURA '{' lista_instrukcji  '}';
lista_instrukcji   : instrukcja+;
instrukcja   :   instrukcja_wyboru
             |   instrukcja_petli
             |   instrukcja_przerwania_petli
             |   instrukcja_kontynuacji_petli
             |   instrukcja_wkroczenia
             |   instrukcja_powrotu
             |   instrukcja_zlozona
             |   instrukcja_prosta
             |   instrukcja_pusta
             |   deklaracja_nazwy;


instrukcja_wyboru   : ('jeśli'|'jesli'|'gdy') '(' wyrazenie ')' instrukcja  ('inaczej'  instrukcja)?;
instrukcja_petli   : 'dopóki' '(' wyrazenie ')' instrukcja;
instrukcja_powrotu   : 'zwróć' wyrazenie? ';';
instrukcja_wkroczenia   : ('zacznij'| 'tu')  NAZWA '(' lista_parametrow_formalnych ')' ('->' obutyp)? instrukcja;
instrukcja_przerwania_petli   : PRZERWIJ ';';
instrukcja_kontynuacji_petli   : KONTYNUUJ ';';
instrukcja_zlozona  : '{'  lista_instrukcji?  '}';
instrukcja_prosta  :   wyrazenie ';';
instrukcja_pusta   : ';';

lista_parametrow_formalnych : (deklaracja_parametru  (',' deklaracja_parametru)* (',' ELIPSA )? )?;
deklaracja_parametru   : deklarator_bez_przypisania przydomkowany_prawotyp| przydomkowany_lewotyp  deklarator_bez_przypisania;
@\newpage@

wyrazenie
          : wyrazenie '(' lista_parametrow_aktualnych ')'                           #wyrazenieWywolanie
          | lewotyp '(' lista_parametrow_aktualnych ')'                             #wyrazenieKonwersja
          | alokacja                                                                #wyrazenieAlokacja
          | dealokacja                                                              #wyrazenieDealokacja
          | wyrazenie '['  wyrazenie  ']'                                           #wyrazenieSelekcjaTablicowa
          | wyrazenie '.' NAZWA                                                     #wyrazenieSelekcjiSkladowej
          | adr='&' wyrazenie                                                       #wyrazenieAdres
          | neg='!' wyrazenie                                                       #wyrazenieNegacja
          | znak=('-'| '+') wyrazenie                                               #wyrazenieZnak
          | <assoc=right> wyrazenie '^' wyrazenie                                   #wyrazeniePoteg
          | wyrazenie mult=('*' | '/' |'%') wyrazenie                               #wyrazenieMult
          | wyrazenie addyt=('+' | '-') wyrazenie                                   #wyrazenieAddyt
          | wyrazenie porownanie=('==' | '!=' | '>' | '<' | '<=' | '>=')wyrazenie   #wyrazeniePorownanie
          | wyrazenie logicz=('&&' | '||')wyrazenie                                 #wyrazenieLogicz
          | <assoc=right> wyrazenie '=' wyrazenie                                   #wyrazeniePrzypisanieZwykle
          | <assoc=right> wyrazenie '^=' wyrazenie                                  #wyrazeniePrzypisaniePoteg
          | <assoc=right> wyrazenie mult=('*=' | '/=' | '%=') wyrazenie             #wyrazeniePrzypisanieMult
          | <assoc=right> wyrazenie addyt=('+=' | '-=') wyrazenie                   #wyrazeniePrzypisanieAddyt
          | NAZWA                                                                   #wyrazenieNazwa
          | literal                                                                 #wyrazenieLiteral
          | '(' wyrazenie ')'                                                       #wyrazenieNawiasy
          ;

alokacja: NOWY wielkosc_alokowanej_tablicy? obutyp;// buf = nowy [128] całk; vs buf = new int[128]; - przestawiony rozmiar przed typ względem C++
wielkosc_alokowanej_tablicy: '[' wyrazenie ']';
dealokacja:ZAPOMNIJ '('wyrazenie')';

lista_parametrow_aktualnych : (wyrazenie  (',' wyrazenie)*)?;

deklaracja_nazwy   :
     deklarator_bez_przypisania prawotyp /*('='wyrazenie)?*/ ';'
     | przydomkowany_lewotyp (deklarator_bez_przypisania|deklarator_zlozony_z_przypisaniem) (',' (deklarator_bez_przypisania|deklarator_zlozony_z_przypisaniem))*  ';';

deklarator_bez_przypisania : NAZWA;
deklarator_zlozony_z_przypisaniem : NAZWA  '='  wyrazenie;

przydomkowany_lewotyp: przydomki lewotyp;
przydomkowany_prawotyp: przydomki prawotyp;
przydomki : (STATYCZN|AUTOMATYCZN|STAL)*;

lewotyp : nazwa_typu (podnosnik_krotnosci_wskaznikowej)*;
prawotyp : (podnosnik_krotnosci_wskaznikowej)* typ_strzalkowy;
obutyp: lewotyp | prawotyp;
podnosnik_krotnosci_wskaznikowej: '^';

typ_strzalkowy: '(' lista_parametrow_strzalkowego ')' '->' obutyp;
lista_parametrow_strzalkowego : (parametr_strzalkowego  (',' parametr_strzalkowego)* (',' ELIPSA )?)?;
parametr_strzalkowego: obutyp;

nazwa_typu:  NAZWA_TYPU | TCALK | TRZECZYW  | TZNAK | TNIC;

literal: literal_atomiczny | literal_tablicowy;
 literal_atomiczny   :  CALK  |  ZMIENN  |  ZNAK_DOSL ;
 literal_tablicowy   : NAPIS_DOSL | PUST;
@\newpage@
// LEKSER

 ELIPSA:'...';
 TNIC : 'Nic';
 TCALK: 'ca'[\u0142]'k'('owit'([yea]|'ych')?)?;
 TRZECZYW: 'rzecz'('yw'('ist'[yea])?)?;
 TZNAK: 'znak';

 PUST: 'pust'[yea];

 STATYCZN: 'statyczn'[yea];
 AUTOMATYCZN : 'automatyczn'[yea];
 STAL : 'sta'[\u0142][yea];

 PROCEDURA: 'proc'| 'procedura';
 NOWY: 'now'[yea];
 ZAPOMNIJ: 'zapomnij';

 PRZERWIJ : 'przerwij';
 KONTYNUUJ:'kontynuuj';

 ZMIENN :  CALK'.'CALK; //zmiennoprzecinkowa liczba
 CALK :   [0-9]+ ;   //zwykła liczba
 ZNAK_DOSL
     :   '\'' ( EscapeSequence | ~('\''|'\\') ) '\''
     ;

 NAPIS_DOSL
     :  '"' ( EscapeSequence | ~('\\'|'"') )* '"'
     ;

 fragment
 EscapeSequence
     :   '\\' ('b'|'t'|'n'|'f'|'r'|'\''|'\\');
//RZECZYWISTE TOKENY, ZAMIENIANE PRZEZ PREPROCESOR
     NAZWA:[\u9670];
     NAZWA_TYPU:[\u967D];
     ID : ([A-Za-z_]|OGONEK)([0-9A-Za-z_]|OGONEK)*;//zob.preprocesor
//TOKENY DZIAŁAJĄCE BEZ PREPROCESORA
//    NAZWA_TYPU :  ([A-Z]|WIELKI_OGONEK)([0-9A-Za-z_]|OGONEK)*;//zaczyna się z dużej litery
//    NAZWA : ([a-z_]|OGONEK)([0-9A-Za-z_]|OGONEK)*;//zaczyna sie z małej litery
//    ID:'$';

//polskie ogonki
 fragment OGONEK : [\u0104\u0105\u0106\u0107\u0118\u0119\u0141-\u0144\u015A\u015B\u0179-\u017C\u00D3\u00F3];
 fragment WIELKI_OGONEK: [\u0104\u0106\u0118\u0141\u0143\u015A\u0179\u017B\u00D3];
 //104,105 - Ąą
 //106,107 - Ćć
 //118,119 - Ęę
 //141,142 - Łł
 //143,144 - Ńń
 //15A,15B - Śś
 //179,17A - Źź
 //17B,17C - Żż
 //D3,F3   - Óó


 LINE_COMMENT : '//' .*? (EOF|'\r'? '\n') -> skip ;
 COMMENT : '/*' .*? '*/' -> skip ;
 WS  :   [ \t\r\n]+ -> skip ; // ignorowanie białych znaków
\end{lstlisting}
\lstset{
    escapechar=|,
    breaklines=true
}
\restoregeometry % Restore the original page margins after the listing

Aby uczynić zadość przyjmowanej szeroko definicji gramatyki bezkontekstowej\cite[str.~158]{hopcroft_automaty}, musimy podać, że zbiorem symboli nieterminalnych są te, zapisywane małymi literami w produkcjach, zbiorem terminali te pisane wielkimi (wedle konwencji generatorów parserów, nie podręczników lingwistyki formalnej, a przynajmniej \cite{hopcroft_automaty}), zaś symbolem startowym jest „program”. 

Jest to język bezkontekstowy, o ile za symbol terminalny przyjmujemy tokeny, gdyż każda produkcja gramatyki ma tylko jeden nieterminal po lewej stronie. Jeśli zaś rozpatrujemy ten język jako ciąg znaków, nie tokenów, wykracza on poza tę klasę, ze względu na kontekstowość mechanizmu odróżniającego tokeny NAZWĄ i NAZWA\_TYPU.

W translatorze istnieje komponent umieszczony pomiędzy lekserem a parserem, działający na strumieniu tokenów i przekształcający tokeny ID według następującego automatu:

\begin{figure}[h]
    \centering
    \includesvg[width=0.6\textwidth]{images/wstep/preprocesor.svg}
    \caption{Automat odpowiadający nieterminalowi deklaracja\_typu, zbierający wpierw nazwy wszystkich nowych typów. ($\Sigma$ : zbiór terminali) }
\end{figure}
Uzyskawszy w ten sposób zbiór nazw wszystkich typów użytkownika wprowadzonych w jednostce translacji (a można je deklarować jedynie globalnie, ze względu na umiejscowienie symbolu deklaracja\_typu), przekształca się każdy token ID albo na token NAZWA\_TYPU, jeśli jego tekst należy do zbioru nazw typów, albo NAZWA w przeciwnym wypadku.

<<Powszechne języki, takie jak C nie są w rzeczywistości zuepłnie bezkontekstowe, a mądre wyznaczenie granic pomiędzy lekserem (z językiem regularnym), parserem (bezkontekstowym), a semantyką (pisaną w języku kompletnym w sensie Turinga), tak aby optymalnie wykorzystać możliwości każdego z tych mechanizmów, jest istotnym zadaniem w pojektowaniu translatora [DRAGON BOOK] ???????? czy gdzies dalej coś o tym jest???>>

Uważny czytelnik może zwrócić też uwagę na drobne niedopatrzenie w tym rozdziale – zapomniano  przetłumaczyć projekt języka na angielski. To odstępstwo od zwyczaju będzie wymagało szerszych wyjaśnień.

\section{Dlaczego po polsku?}
1. Od dawna nikt nie próbował
Pierwszym uzasadnieniem dla próby skonstruowania języka programowania „po polsku”, jest to , że obecnie żaden taki język nie istnieje. Każdy, kto przeprowadził kwerendę w tym zakresie, zorientuje się szybko, że publicznie wiadomo o dokładnie jednej próbie stworzenia języka ogólnego przeznaczenia z polskim substratem.
W latach 60 w Zakładzie Aparatów Matematycznych Polskiej Akademii nauk, dla serii rodzimych komputerów(choć inspirowanych IBM 701), stworzono System Automatycznego Kodowania, nazywany częściej językiem SAKO. Było to niewątpliwie ciekawe przedsięwzięcie, jako jedyne w historii proponujące odpowiedź na główną przeszkodę w pisaniu programów „po polsku” - konieczność pisania pomijając odmianę wyrazów w naszym języku. W SAKO zaproponowano proste, lecz akceptowalne wtedy rozwiązanie – identyfikowanie symbolu tylko na podstawie kilku pierwszych jego liter (co pozwala na różne końcówki). Wobec liczącego w ledwie tysiącach słów maszynowych rozmiaru pamięci tych komputerów i konsekwentnie rozmiarów pisanych programów, nie stanowiło to praktycznego problemu. 
Niektóre aspekty tego języka są niezwykle interesujące, czego dobrym przykładem jest obsługa przepełnień, do czego przeznaczono specjalną instrukcję:
BYL NADMIAR , np.:
GDY BYL NADMIAR: 1A, INACZEJ 3 
To z pewnością więcej, niż współczesny mu FORTRAN, czy C, które „rozprawiają się” z problemem, orzekając w swych standardach, że powoduje on zawsze niezdefiniowane zachowanie (undefined behaviour) i przenosząc całą odpowiedzialność na programistę.
Wymarł jednak ten język, wraz z komputerami zawierającymi rodzimą myśl techniczną i od tego momentu programuje się wyłącznie po angielsku.
Poszukując innych przykładów, nie znajdziemy praktycznie nic, prócz wąskich języków domenowych. Istniała polska wersja pogramu MS Logo (Logomacja), gdzie przetłumaczono rozkazy podawane figuratywnemu żółwiowi na język polski, a jako, że zawierał on również instrukcj warunkowe, pętle i możliwość definiowania zmiennych, to przyznać mu należy cechę kompletności Turinga. Można cierpko stwierdzić, że oprócz kilku narzędzi edukacyjnych, okazjonalnie uzyskujących polskie tłumaczenia, jedynym językiem (w dodatku czysto funkcyjnym) opartym polskim substracie, posiadających setki tysięcy, a może miliony użytkowników w naszym kraju jest… Excel.

Używanie jedynie czterech (czasem trzech) pierwszych liter symbolu do jego identyfikacji, byłoby przy dzisiejszej objętości programów niepraktyczne. Niniejsza praca nie podejmie wszak postawionego przez SAKO pytania o właściwą formę języka programowania „po polsku”, lub opartego na języku fleksyjnym w ogóle (nie aglutynacyjnym, jak angielski) – wykracza to poza jej zakres, którego domknięciem w najlepszym wypadku będzie skonstruowanie translatora posiadającego techniczne możliwości obsługi języka tego rodzaju, na razie ograniczając wszak inwencje składniową do minimum, w pragmatycznym celu zapewnienia w pierwszej kolejności niezawodności i odpowiedniej architektonicznej bazy, korzystając z której można by język ów rozbudować.

2. Polskie dziedziny problemów
Choć zdarzają się przypadki niemal religijnego przywiązania do jednego języka, większość programistów doskonale zdaje sobie sprawę, że do różnych problemów, najodpowiedniejsze są różne języki. Istnieją też problemy, które opisywane są całkowicie, lub prawie całkowicie po polsku. Spójrzmy na przykład xml-a będącego dokumentem przesłanym na skrzynkę e-doręczeń. 
[]
Kolejnym oczywistym przypadkiem są reguły podatkowe, których mnogość w tym kraju jest nam wszystkim dobrze znana i zapewnia chociażby stałą niszę rynkową dla miejscowego oprogramowania ERP dla biznesu, gdyż produkty zachodnie ich nie uwzględniają. Nie musimy wszakże odwoływać się do przykładów zależnych tak silnie od arbitralnych uwarunkowań. Czasem dziedzina problemu naturalnie nie zawiera języka angielskiego.
Usłyszałem kiedyś następujące pytanie: „Cykl czytań mszalnych liczy cztery lata, czy podczas tego okresu odczytuje się cały tekst Biblii?”. Dysponując tekstem oraz siglami czytań na każdy dzień, o co nietrudno, da się to sprawdzić, pisząc prosty skrypt. Można przypuszczać, że gdyby napisano go w Pythonie mógłby zawierać fragment podobny do zamieszczonego poniżej:
\begin{lstlisting}
for day, siglum in cycle:
	for verse in split_verses(siglum):
		verses[verse] |= {day}    
\end{lstlisting}

Obiekty dziedziny problemu noszą w istocie polskie nazwy i czy dodawaniem do ich zbioru języka angielskiego nie jest „mnożeniem bytów ponad potrzebę?”
\begin{lstlisting}
dla dnia, siglum spośród cyklu:
	dla wersu spośród wydzielonych_wersów(z: siglum):
		wersy[wers] |= {dzień}
\end{lstlisting}
Sama konieczność używania języka angielskiego, nie stanowi już chyba dla nikogo związanego z branżą większego problemu, niesie jednak ze sobą ukryte koszty, ilekroć stosujemy go do opisu rodzimych problemów. Moje doświadczenie zawodowe zawiera pracę nad systemem, który choć występuje w wersji angielskiej w aplikacjach mobilnych, to działa wyłącznie na terytorium Polski i posługuje się polską nomenklaturą w dziedzinie bytów i procesów biznesowych, we wszystkich miejscach – od panelu przeglądarkowego używanego przez operatorów, przez dokumentację aż do nieoficjalnych rozmów – wszędzie, prócz samego kodu. Przetłumaczenie podstawowych nazw encji, rzecz jasna nie pochłonęło wiele wysiłku, poziom trudności wzrósł na przykład dla nazewnictwa księgowego. Zapewne nie istnieje dokładny semantyczny odpowiednik nazw dokumentów KP/KW (kasa przyjęła, wydała) w języku angielskim, jak i wielu innych, zwłaszcza jeśli odnoszą się do systemu prawnego. Inna natura prawa anglosaskiego względem systemów kodeksowych, do których nas należy, powoduje, że odpowiedniość pewnych terminów może być nawet złudna. Równie wiele czasu zaprzepaścić można szukając rozsądnego tłumaczenia neologizmu będącego dziełem działu marketingowego, choć działającym zupełnie realnie i potrzebującego osobnej encji w bazie danych i kodzie serwerowym. W takim wypdaku dychotomia nomenklatury jawi się jako skryty, lecz stale obecny koszt, 

Przypuszczam również, że mogą istnieć jednostki niekiedy znużone nieustanną wszechobecnością angielszczyzny w pracy informatyka, które z chęcią, dla odmiany napisałyby niewielki skrypt, bądź fragment aplikacji po polsku, lub zwyczajnie zaciekawione by były, jak taki język „wyglądał” i jak sprawowałby się w praktyce.


	\chapter{Implementcja translatora}
\section{Parser}
Istnieje ścisła zależność pomiędzy każdym typem języków w hierarchii Chomskiego, ich gramatykami i automatami rozpoznającymi napisy w tych językach.[][]
\begin{center}
\begin{tabular}{|c|c|c|}
\hline
\textbf{Stopień w hierarchii} & \textbf{Gramatyka} & \textbf{Automat} \\ \hline
0 & Struktur frazowych (nieograniczona) & Maszyna Turinga \\ \hline
1 & Kontekstowa & Liniowo ograniczony\\ \hline
2 & Bezkontekstowa & Ze stosem\\ \hline
3 & Regularna & Skończenie stanowy\\ \hline
\end{tabular}
\end{center}

Dla automatycznej analizy składniowej użyteczne są dwa dolne stopnie tej hierarchii. Analizę leksykalną przeprowadza się definiując tokeny z użyciem języków regularnych, (Chociaż napisanie efektywnego leksera jest daleko trudniejsze niż samo rozpoznawanie wyrażeń regularnych).
Samo egzystencjalne twierdzenie o istnieniu automatu ze stosem rozpoznającego język bezkontekstowy, niewielką jest pociechą dla piszącego jego translator – potrzebny jest konkretny sposób jego konstrukcji. Nie jest to zagadnienie trywialne i poświęcono mu wiele uwagi w pierwszych dekadach rozwoju języków wysokopoziomowych. Nie jest możliwe, w tekście tego rodzaju, choćby naszkicowanie kształtów wielkiego gmachu lingwistyki formalnej, ani nawet teorii parsingu. Powiedziane zostanie zaledwie tyle, by umotywować wybór narzędzia do analizy składniowej.

Istnieje kilka algorytmów konstrukcji automatu dokonującego rozbioru zdań w języku opisywanym zadaną gramatyką bezkontekstową. Automat taki nazywany bywa również parserem i miano ta odpowiada jego naturze – bo istotnie wydziela części (łac. - pars) zdania, konstruując drzewo składniowe, gdzie liśćmi są terminale, gałęziami symbole nieterminalne, a korzeniem symbol startowy (choć w praktyce najłatwiej stworzyć parser jedynie rozpoznajacy, czy dany napis należy do języka, czy nie, zachowanie parsera zawiera informację o strukturze składniowej zdania, lecz aby ją wydobyć i skonstruować drzewo, potrzebna jest dodatkowa praca).[rysunek dodać przykładowy]
Najprościej wyjaśnić można działanie parsera rekursywnie zstępującego (recursive descent). Konstrukcja jego jest w zasadzie trywialna. Załóżmy, że mamy produkcje gramatyki:

\begin{lstlisting}[numbers=left]
instr: 'jeśli' wyr 'to' instr 'koniec';
instr: 'dopóki' wyr 'to' instr 'koniec';
instr: lista_instr;
lista_instr: instr ';' lista_istr;
lista_instr: |$\epsilon$|; (pusta prawa strona produkcji)
\end{lstlisting}

Wystarczy, że zaopatrzymy się w analizator leksykalny z funkcją nast(), która zwraca następny token, dopasuj(), która konsumuje ów ze strumienia wejściowego, to możemy niemal wprost „przepisać” gramatykę na procedury parsera:

\begin{lstlisting}
    void instr(){
	switch(nast())
	case 'jeśli': wyr(); to(); instr(); koniec(); break; 
            (zob. produkcję (1))
	case 'dopóki': wyr() to(); instr(); koniec(); break; 
            (zob. produkcję (2))
	default ... - przepisać z DRAGON BOOK
}
\end{lstlisting}
Nie skonstruujemy wszak w ten sposób parsera wyrażeń arytmetycznych:
\begin{lstlisting}
wyr: wyr * wyr| wyr + wyr;…
void wyr(){
wyr(); ‘+’ wyr()...
}
\end{lstlisting}
gdyż oznaczałoby to nieskończoną rekursję. (Rekursja lewostronna jest ogólnie zakazana dla parserów zstępujących)

Podobnie działa tabelaryczny parser LL, z tą różnicą, że ma formę jawnego automatu ze stosem, a nie zbioru procedur. Istnieją różne odmiany algorytmu generacji zbioru stanów takich parserów – kanoniczne LL(k) są w stanie rozpoznawać szersze klasy języków (im dalej pozwala im się podglądać wprzód, choć rośnie wtedy rozmiar zbioru stanów), SLL z kolei słabszy jest nawet od LL(1), lecz bardzo łatwy w konstrukcji.
Ograniczenie parsera rekursywnie schodzącego, lub LL da się do pewnego stopnia omijać i niwelować, jednakże dowiedziono, że klasa języków LL(k) – parsowanych od lewej do prawej strony, zstępująco (od symbolu startowego do terminali), jest węższa niż LR(k) – wstępujących.
Donald Knuth 1965 w swoim artykule „Parsowanie od lewej do prawej” przedstawił algorytm konstrukcji parsera LR – wstępującego – który jako pierwszy algorytm rozbioru posiadał gwarancję działania w czasie liniowym względem długości wejścia. Dostrzeżono powszechnie wielki potencjał w parserach LR,  choć ich tabele (zbiory stanów), nawet dla LL(2) okazywały się zbyt obszerne względem dostępnej pamięci[]. Dość szybko, w roku 19XX de Roemer zaproponował efektywny sposób ich kompresji[]. Ta odmiana algorytmu – LALR, została zastosowana w generatorze parserów yacc – gdyż generowanie tabel według algorytmów można zautomatyzować i programista musi wtedy stworzyć jedynie gramatykę, a narzędzie przekształci ją w parser. W 197X w kompilatorze C XXX zmieniono parser na generowany przez yacc, w miejsce pisanego ręcznie rekursywnie schodzącego. Na długie lata generowane parsery LALR stały się standardem, a para uniksowych programów lex-yacc, podstawą zarówno wielu narzędzi – np. awk, jak i poważnych jezyków – perl.
Z powodu tej ugruntowanej pozycji LALR, wielkim zaskoczeniem dla postronnych mogła być decyzja projektu gcc, gdzie w 2006 roku wymieniono parser z powrotem na ręcznie pisany, rekursywnie schodzący. Przyczyny tego „regresu” można dopatrywać się w zwiększonych wymaganiach użytkowników względem komunikatów o błędach. W ręcznie pisanym parserze, można umieścić dowolnie złożoną obsługę błędów. Zarządzanie zachowaniem generowanego parsera LR jest natomiast trudniejsze, oznacza manipulowanie stanem stosu symboli i odpowiednio dużej ilości niezbędnych poprawek, można stwierdzić, że prościej byłoby taki parser napisać ręcznie. Ponadto LR, będąc parserem wstępującym, nie posiada informacji, wewnątrz jakiego symbolu nieterminalnego znajduje się analizowana fraza, w przeciwieństwie do LL, który „schodzi” od symbolu startowego gramatyki przez kolejnie nieterminale. Ta sama własność, która pozwala LR rozpoznawać szerszą klasę języków, pozbawia go jednocześnie części informacji przydatnych w konstruowaniu zrozumiałych dla człowieka komunikatów o błędach.[parsing timeline „LR fast but stupid”]

Teoretycznie jeszcze lepsze komunikaty może zapewnić parser działający wedle algorytmu Earleya[]. W przeciwieństwie do LL i LR, rozpoznaje on wszystkie języki bezkontekstowe, również te niejednoznaczne, zwracając wszystkie możliwe rozbiory. Ceną za taką ogólność jest jednak wydajność. Oryginalny algorytm z 197X roku miał złożoność O(n4??) i z tego powodu nie zyskał szerszej popularności, co zrozumiałe, ze względu ówczesne na ograniczenia sprzętowe. Joop Leo w 1997?? zoptymalizował wszak jego działanie dla rekursji prawostronnych, uzyskując pesymistyczna złożoność O(n??), a w 2002 roku poprawiono drobny błąd dla produkcji o zerowej długości. Wskutek tych poprawek, zmodyfikowany algorytm Earleya, ma praktyczną złożoność O(n) dla całek klasy LR(k) – taką samą jak algorytm Knutha, i zwalnia do O(n2??) dla niektórych dowolnych gramatyk bezkontekstowych. W ostatnich latach więc zdobywa coraz większą popularność w zastosowaniach praktycznych, czego przykładem może być biblioteka nearley[].

Istnieją tez alternatywne podejścia – parsowanie Pratta wyrażeń z operatorami, któremu zawdzięczamy powszechne w specyfikacjach języków tabele z poziomami precedencji operatorów, czy inne podejście nieoparte na gramatykach Chomskiego – GEP. Rzeczywistość jest oczywiście dużo bardziej złożona, niż tu opisano i w praktycznych zastosowaniach mówi się również o LL(*), LR(*), GLR, LRR i wielu innych rozwiązaniach pośrednich.

Jednym z nich jest ANTLR. Teoretycznie jest predykcyjnym parserem rekursywnie schodzącym, lecz do wyboru ścieżki używa ATN (augmented translation networks – wzbogaconych sieci tłumaczeniowych) – konceptu właściwego początkowo parserowi GLR – wstepującemu, tutaj zastosowanymi dla analizaatora zstepującego.
Ogólnie procedura generowana przez antlr dla nieterminala wyglada nastepująco:
\begin{lstlisting}
...
adaptivePredict(tok)
...
\end{lstlisting}
Działa więc on jak zwykły parser predykcyjny, nie używając jednak bezpośrednio podglądu tokenów, lecz skomplikowanego mechanizmu, generującego w locie automaty skończenie stanowe. Teoretycznie charakteryzuje się pesymistyczną złożonością O(n4), lecz w praktyce generuje jedne z szybszych parserów i jest powszechnie stosowany. Autor tego tekstu, pracując nad projektami w Javie, zupełnie z pozoru niezwiązynymi z zagadnieniami parsingu, odnajdywał wielokrotnie antlr na liscie zależności – używają go biblioteki do parsowania html, json i inne powszechnie potrzebne.
...najfajniejsze… -automatyczne usuwanie rekursji
… gramartyki naturalne… niepotrzebne atryuty dziedziczone, brak akcji – to później

    \chapter{Wnioski}
Ta praca jest za długa!!
	
	% itd.
	% \appendix
	% \include{dodatekA}
	% \include{dodatekB}
	% itd.
	
	\printbibliography

\end{document}
